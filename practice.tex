\begin{center}\section{Описание экспериментов}\end{center}
\subsection{Эксперимент 1. Проверка моделей юстировки камер: аффинной (A), с учетом дисторсии 3 порядка (D3), с учетом дисторсии 3 порядка (D35)}
% Здесь краткое описание эксперимента
В ходе эксперимента вычислялись параметры линеаризованной связи координат левого и правого кадра стереопары. Для каждой из трех моделей виртуальной юстировки решалась задача теории измерительно-вычислительных систем \eqref{scheme_measure} $\xi = A g + \nu$. Ее решение:
$$
    g = (a, b, c, d, e, f, \eps_1, \eps_2, \mu_1, \mu_2) = A^- \xi,
$$
где $A^-$ --- псевдообратная матрица (считается с помощью встроенной функции пакета NumPy numpy.linalg.pinv).

Результаты вычисления параметров связи для всех трех моделей можно видеть в таблице \ref{table:coeffs_reliability}.
Также, в последнем столбце указано значение надежности($\tau$) для найденных коэффициентов.

% Далее, в первую очередь избавляемся от дисторсий на левом и правом изображениях (переписываем старые точки в новые места):
% \begin{python}
% nleft[round(i - eps1 * zx1 - mu1 * wx1), round(j - eps1 * zy1 - mu1 * wy1)] = left[i, j]
% nright[round(i - eps2 * zx2 - mu2 * wx2), round(j - eps2 * zy2 - mu2 * wy2)] = right[i, j]
% \end{python}

% Затем, применяем аффинное преобразование с найденными коэффициентами
% $(a, b, c, d, e, f)$, используя бикубическую интерполяцию ( функция PIL.Image.transform ).

% Здесь оцененные коэффициенты: только аффинные / с дисторсией
\begin{table}[H]
\centering
\caption{Эксперимент 1. Параметры связи координат и надежность модели ($\tau$)}
\label{table:coeffs_reliability}

\begin{tabular}{|l|l|l|l|l|l|}
\hline
\multicolumn{6}{|l|}{Только аффинное преобразование}\\ \hline
a     & b     & c       & d     & e      & f     \\ \hline
0.998 & 0.014 & -0.009  & 0.996 & 234 & 55 \\ \hline

$\eps_1, 10^{-8}$ & $\eps_2, 10^{-8}$ & $\mu_1, 10^{-14}$ & $\mu_2, 10^{-14}$ & $\tau$ & \\ \hline
-                 & -                 & -                 & -                 & 0.08   & \\ \hline

\hline
\multicolumn{6}{|l|}{Учет дисторсий 3го порядка}\\ \hline
a     & b     & c      & d     & e   & f  \\ \hline
0.995 & 0.014 & -0.010 & 0.995 & 234 & 56 \\ \hline

$\eps_1, 10^{-8}$ & $\eps_2, 10^{-8}$ & $\mu_1, 10^{-14}$ & $\mu_2, 10^{-14}$ & $\tau$ & \\ \hline
3.38              & 4.03              & -                 & -                 & 0.32   & \\ \hline

\hline
\multicolumn{6}{|l|}{Учет дисторсий 3 и 5 порядка}\\ \hline
a     & b     & c      & d     & e   & f  \\ \hline
0.988 & 0.014 & -0.010 & 0.988 & 245 & 64 \\ \hline

$\eps_1, 10^{-8}$ & $\eps_2, 10^{-8}$ & $\mu_1, 10^{-14}$ & $\mu_2, 10^{-14}$ & $\tau$ & \\ \hline
0.93              & 5.42              & 6.91              & -2.12             & 0.19   & \\ \hline
\end{tabular}
\end{table}

В таблице \ref{table:coeffs_reliability} можно заметить, что коэффициенты $a, b, c, d$ матрицы поворота близки к аналогичным от единичной матрицы. Значения дисторсных коэффициентов $\eps_1, \eps_2, \mu_1, \mu_2$ --- много меньше 1. Коэффициенты для дисторсии 5-го порядка имеют порядок $10^{-14}$, что близко к границе точности чисел типа double(число с плавающей запятой двойной точности, 64 бит) --- $10^{-15}$, которые использовалось в программе. Поэтому, нельзя исключать влияние ошибок машинного округления для модели D35.

Также на рисунках \ref{fig:align_a}, \ref{fig:align_d3}, \ref{fig:align_d35} можно увидеть итоговое положение звезд после процедуры виртуальной юстировки изображений. Можно заметить, что модели хорошо совмещают точки в центральной части. Однако, для совмещения периферийных звезд этих моделей оказывается недостаточно.

\begin{figure}[H]
  \centering
  \includegraphics[width=0.8\linewidth]{images/mag_diss/20161122-191517-359_000.png}
  \caption{Совмещение звезд. A: Только аффинное преобразование}
  \label{fig:align_a}
\end{figure}%

\begin{figure}[H]
  \centering
  \includegraphics[width=0.8\linewidth]{images/mag_diss/20161122-191517-359_030.png}
  \caption{Совмещение звезд. D3: С учетом дисторсии 3го порядка}
  \label{fig:align_d3}
\end{figure}

\begin{figure}[H]
  \centering
  \includegraphics[width=0.8\linewidth]{images/mag_diss/20161122-191517-359_035.png}
  \caption{Совмещение звезд. D35: С учетом дисторсии 3го, 5го порядков}
  \label{fig:align_d35}
\end{figure}

% % Результаты
% Можно заметить (иллюстрации на Рис. (\ref{fig:stars_center}, \ref{fig:stars_edge2}, \ref{fig:stars_edge3}) и Табл. \ref{tab:aff_dist_coeffs}), что:
% \begin{itemize}
%     \item Вблизи центра совмещение стало немного хуже, на краях --- значительно лучше.
%     \item Матрица поворота $ \begin{pmatrix} a & b \\ c & d \end{pmatrix} \approx \begin{pmatrix} 1 & 0 \\ 0 & 1 \end{pmatrix},$ т.е. камеры уже довольно хорошо съюстированы вручную.
%     \item $\eps_1 \approx \eps_2$, т.е. значения коэффициентов дисторсии на двух камерах согласуются.
% \end{itemize}

%
%%%%%%%%%%%
%

\newpage
\subsection{Эксперимент 2. Зависимость надежности модели от размера области выбора звезд для юстировки}
% % Здесь краткое описание эксперимента
В ходе этого эксперимента звезды, используемые для виртуальной юстировки выбирались следующим образом. Для вычисления параметров моделей A, D3, D35 использовались лишь те звезды, которые попадали в двух круги радиуса $R$, с центрами в соответствующих центральных точках левого и правого кадров. Затем вычислялось значение надежности для найденных коэффициентов по формуле:
\begin{equation}
  \label{eq:reliability}
  \tau_k(\xi) = \frac{ (2N - m_k) \sigma^2 }
                     { \| (I - A_k A_k^-) \xi \|^2 }
  ,
\end{equation}
где $k \in \{ A, D3, D35 \}$, $ m_k \in \{6, 8, 10\} $, $A_k$ --- матрица связи из \eqref{scheme_measure} для соответствующей модели $k$,
$\sigma$ --- корень из дисперсии вектора аддитивного шума.

Радиус центральной части увеличивался и вся процедура повторялась.

На \ref{fig:reliability_on_area} изображен итоговый график надежности для трех моделей. По горизонтальной оси отложено количество звезд, попавших в центральную область. Можно заметить, что надежность убывает с расширением центральной области. Следует отметить, что наилучшие результаты показывает модель, учитывающая дисторсию 3го порядка (D3).

% График \tau(n)
\begin{figure}[H]
  \centering
  \includegraphics[width=1\linewidth]{images/mag_diss/plot--reliability--on--stars_num.png}
  \caption{Зависимость надежности модели от размера области выбора звезд для юстировки. По горизонтальной оси отложено количество звезд, попадающих в центральную область (используемую для юстировки)}
  \label{fig:reliability_on_area}
\end{figure}

%
%%%%%%%%%%%
%

\newpage
\subsection{Эксперимент 3. Оценка значений высоты облаков при различных моделях юстировки}
% Здесь краткое описание эксперимента
В ходе эксперимента на предварительно съюстированной стереопаре решалась задача поиска фрагмента на изображении.
Затем, из найденного пиксельного сдвига между фрагментами вычислялось значение расстояния до объекта (облака) по формуле \eqref{eq:D_final}:
\begin{equation*}
  D = \frac{ B \xi_0 }
           { 2 tg \frac{\varphi_0}{2} | \xi_1 - \xi_2 | }
\end{equation*}

Ошибка высоты вычислялась по формуле
\begin{equation*}
    \Delta D =  \sqrt{ (\frac{\partial D}{\partial B}\Delta B)^2 + (\frac{\partial D}{\partial \varphi_0}\Delta \varphi_0)^2 + 2(\frac{\partial D}{\partial \xi_1}\Delta \xi_1)^2}
\end{equation*}

Задача поиска фрагмента на изображении решалась методами морфологического анализа изображений. При этом была использована оптимизация алгоритма поиска за счет вычисления сверток функций через быстрое преобразование Фурье. Таким образом, за счет увеличения количества используемой оперативной памяти, сложность алгоритма поиска уменьшилась с $O(N^2)$ до $O(N log(N))$, где $N$ --- характерный линейный размер изображения в пикселях.

\begin{figure}[H]
  \centering
  \includegraphics[width=\linewidth]{images/mag_diss/plot_20160831-154449-796_50.png}
  \caption{Сравнение результатов оценки высоты нижней границы облачности,
  полученных разработанной системой со значением, полученным лазерным
  дальномером (красная линия)}
  \label{fig:laser_cloud_height}
\end{figure}

На рисунке \ref{fig:laser_cloud_height} можно увидеть оценки высот облаков с их погрешностями для одной стереопары.
По вертикальной оси отложены значения высоты, по горизонтальной измерения упорядочены по интегральной яркости искомого фрагмента.
Красная горизонтальная линия представляет значение высоты, измеренное лазерным дальномером (ЛПР-1) в момент, близкий ко времени съемки стереопары.
Значения вблизи нуля высоты --- ошибки алгоритма поиска фрагмента.

\begin{table}[H]
\centering
\caption{Результаты оценки высоты облаков для различных фрагментов на 3-х стереопарах}
\label{tab:height}
\begin{tabular}{|l|l|l|l|l|}
\hline
\textnumero        & Модель A, м   & Модель D3, м & Модель D35, м & Значение дальномера, м \\ \hline
\multirow{3}{*}{1} & $2716 \pm 440$  & $2716 \pm 440$ & $4345 \pm 977$  & \multirow{3}{*}{1600 --- 1700}\\
                   & $2716 \pm 440$  & $3104 \pm 546$ & $3621 \pm 709$  &                               \\
                   & $2897 \pm 488$  & $2716 \pm 440$ & $3104 \pm 546$  &                               \\ \hline
%
\multirow{3}{*}{2} & $2556 \pm 399$  & $2716 \pm 440$ & $4828 \pm 118$2 & \multirow{3}{*}{2300 --- 3200}\\
                   & $3104 \pm 546$  & $3342 \pm 619$ & $4828 \pm 118$2 &                               \\
                   & $2716 \pm 440$  & $2897 \pm 488$ & $4345 \pm 977$  &                               \\ \hline
%
\multirow{6}{*}{3} & $1889 \pm 254$  & $1975 \pm 271$ & $3104 \pm 546$  & \multirow{6}{*}{2200}         \\
                   & $1889 \pm 254$  & $2069 \pm 290$ & $2414 \pm 365$  &                               \\
                   & $1889 \pm 254$  & $1810 \pm 240$ & $2414 \pm 365$  &                               \\
                   & $1889 \pm 254$  & $1889 \pm 254$ & $2172 \pm 312$  &                               \\
                   & $2069 \pm 290$  & $2287 \pm 337$ & $3342 \pm 619$  &                               \\
                   & $1975 \pm 271$  & $2069 \pm 290$ & $2414 \pm 365$  &                               \\ \hline
\end{tabular}
\end{table}

В таблице \ref{tab:height} приведены оценки высот с погрешностями, полученные с помощью разработанной системы. Для 3х стереопар (1, 2, 3) были оценены значения высот по нескольким фрагментам для 3х моделей (A, D3, D35) и проведено сравнение со значениями лазерного дальномера.
Лучшую согласованность с данными дальномера показали модели A, D3.