\begin{center}\section{Описание экспериментов}\end{center}
\subsection{Эксперимент 1. Проверка моделей юстировки камер: аффинной (A), с учетом дисторсии 3 порядка (D3), с учетом дисторсии 3 порядка (D35)}
% Здесь краткое описание эксперимента
% Имеется два изображения звездного неба, снятые одновременно двумя идентичными камерами, оси которых параллельны и направлены в зенит --- стереопара (с разрешением 3072x2304). На этих кадрах выделяются точки, соответствующие одним и тем же звездам (в примере на Рис. (\ref{fig:stars_center}, \ref{fig:stars_edge2}, \ref{fig:stars_edge3}) использовалась 41 пара координат соответствующих звезд). Далее изображения <<перекрашиваются>> (чтобы можно было анализировать при наложении двух кадров) по правилам:
% \begin{itemize}
%     \item Если левое изображение, то оставляем только красный канал (только R из RGB), другие два обнуляем.
%     \item Если правое изображение, то оставляем только зеленый канал (только G из RGB), другие два обнуляем.
% \end{itemize}
% Из схемы измерений \eqref{scheme_measure} $\xi = A g + \nu$ получаем решение
% $$
%     g = (a, b, c, d, e, f, \eps_1, \eps_2) = A^- \xi,
% $$
% псевдообратную матрицу считаем с помощью встроенной функции пакета NumPy (numpy.linalg.pinv).

% Далее, в первую очередь избавляемся от дисторсий на левом и правом изображениях (переписываем старые точки в новые места):
% \begin{python}
% nleft[round(i - eps_1*zx1), round(j - eps_1*zy1)] = left[i, j]
% nright[round(i - eps_2*zx2), round(j - eps_2*zy2)] = right[i, j]
% \end{python}

% Затем, применяем аффинное преобразование с найденными коэффициентами
% $(a, b, c, d, e, f)$, используя бикубическую интерполяцию ( функция PIL.Image.transform ).

% Накладываем два изображения, инвертируем цвета.

% Здесь оцененные коэффициенты: только аффинные / с дисторсией
\begin{table}[H]
\centering
\caption{Эксперимент 1. Параметры связи координат и надежность модели ($\tau$)}
\label{tab:aff_dist_coeffs}

\begin{tabular}{|l|l|l|l|l|l|}
\hline
\multicolumn{6}{|l|}{Только аффинное преобразование}\\ \hline
a     & b     & c       & d     & e      & f     \\ \hline
0.998 & 0.014 & -0.009  & 0.996 & 234 & 55 \\ \hline

$\eps_1, 10^{-8}$ & $\eps_2, 10^{-8}$ & $\mu_1, 10^{-14}$ & $\mu_2, 10^{-14}$ & $\tau$ & \\ \hline
-                 & -                 & -                 & -                 & 0.08   & \\ \hline

\hline
\multicolumn{6}{|l|}{Учет дисторсий 3го порядка}\\ \hline
a     & b     & c      & d     & e   & f  \\ \hline
0.995 & 0.014 & -0.010 & 0.995 & 234 & 56 \\ \hline

$\eps_1, 10^{-8}$ & $\eps_2, 10^{-8}$ & $\mu_1, 10^{-14}$ & $\mu_2, 10^{-14}$ & $\tau$ & \\ \hline
3.38              & 4.03              & -                 & -                 & 0.32   & \\ \hline

\hline
\multicolumn{6}{|l|}{Учет дисторсий 3 и 5 порядка}\\ \hline
a     & b     & c      & d     & e   & f  \\ \hline
0.988 & 0.014 & -0.010 & 0.988 & 245 & 64 \\ \hline

$\eps_1, 10^{-8}$ & $\eps_2, 10^{-8}$ & $\mu_1, 10^{-14}$ & $\mu_2, 10^{-14}$ & $\tau$ & \\ \hline
0.93              & 5.42              & 6.91              & -2.12             & 0.19   & \\ \hline
\end{tabular}
\end{table}

\begin{figure}[H]
  \centering
  \includegraphics[width=0.8\linewidth]{images/mag_diss/20161122-191517-359_000.png}
  \caption{Совмещение звезд. A: Только аффинное преобразование}
  \label{fig:center_aff}
\end{figure}%


\begin{figure}[H]
  \centering
  \includegraphics[width=0.8\linewidth]{images/mag_diss/20161122-191517-359_030.png}
  \caption{Совмещение звезд. D3: С учетом дисторсии 3го порядка}
  \label{fig:center_dist}
\end{figure}

\begin{figure}[H]
  \centering
  \includegraphics[width=0.8\linewidth]{images/mag_diss/20161122-191517-359_035.png}
  \caption{Совмещение звезд. D35: С учетом дисторсии 3го, 5го порядков}
  \label{fig:center_dist35}
\end{figure}

% % Результаты
% Можно заметить (иллюстрации на Рис. (\ref{fig:stars_center}, \ref{fig:stars_edge2}, \ref{fig:stars_edge3}) и Табл. \ref{tab:aff_dist_coeffs}), что:
% \begin{itemize}
%     \item Вблизи центра совмещение стало немного хуже, на краях --- значительно лучше.
%     \item Матрица поворота $ \begin{pmatrix} a & b \\ c & d \end{pmatrix} \approx \begin{pmatrix} 1 & 0 \\ 0 & 1 \end{pmatrix},$ т.е. камеры уже довольно хорошо съюстированы вручную.
%     \item $\eps_1 \approx \eps_2$, т.е. значения коэффициентов дисторсии на двух камерах согласуются.
% \end{itemize}


% \begin{figure}[H]
% \begin{subfigure}{0.5\textwidth}
%   \centering
%   \includegraphics[width=\linewidth]{images/exp1_stars_dist_aff/edge2_aff_559x450_inv.png}
%   \caption{Только аффинное преобразование}
%   \label{fig:edge2_aff}
% \end{subfigure}%
% \begin{subfigure}{0.5\textwidth}
%   \centering
%   \includegraphics[width=\linewidth]{images/exp1_stars_dist_aff/edge2_dist_559x450_inv.png}
%   \caption{Учет дисторсий + аффинное преобразование}
%   \label{fig:edge2_dist}
% \end{subfigure}
% \caption{Краевой фрагмент №2 совмещеннных фотографий звездного неба}
% \label{fig:stars_edge2}
% \end{figure}


% \begin{figure}[H]
% \begin{subfigure}{0.5\textwidth}
%   \centering
%   \includegraphics[width=\linewidth]{images/exp1_stars_dist_aff/edge3_aff_296x296_inv.png}
%   \caption{Только аффинное преобразование}
%   \label{fig:edge3_aff}
% \end{subfigure}%
% \begin{subfigure}{0.5\textwidth}
%   \centering
%   \includegraphics[width=\linewidth]{images/exp1_stars_dist_aff/edge3_dist_296x296_inv.png}
%   \caption{Учет дисторсий + аффинное преобразование}
%   \label{fig:edge3_dist}
% \end{subfigure}
% \caption{Краевой фрагмент №3 совмещеннных фотографий звездного неба}
% \label{fig:stars_edge3}
% \end{figure}

%
%%%%%%%%%%%
%

\newpage
\subsection{Эксперимент 2. Зависимость надежности модели от размера области выбора звезд для юстировки}
% % Здесь краткое описание эксперимента
% Для анализа была выбрана гладкая функция двух переменных $f(x, y) = \exp(-x^2-y^2)$. В области $[-1; 1] \times [-1; 1]$ были вычислены значения функции в точках равномерной прямоугольной сетки 10x10 и созданы изображения (10x10 пикселей) оригинальной и сдвинутой на $(dx, dy)$ гауссоид (Рис. \ref{fig:subpix_function}).

% График \tau(n)
\begin{figure}[H]
  \centering
  \includegraphics[width=1\linewidth]{images/mag_diss/plot--reliability--on--stars_num.png}
  \caption{Зависимость надежности модели от размера области выбора звезд для юстировки. По горизонтальной оси отложено количество звезд, попадающих в центральную область (используемую для юстировки)}
  \label{fig:subpix_function}
\end{figure}

% Далее, значения параметров сдвига $g = (\wtil{k}, \wtil{b}, \delta_x, \delta_y)$ оцениваются при решении задачи \eqref{eq:min_problem}:
% $$
%     \Big\| Ag - \xi \Big\|^2 \rightarrow \min_g
% $$

% Были проведены эксперименты с различными значениями сдвигов (при фиксированных $(k, b)$, Табл. \ref{tab:expon_dxdy}). Результаты:
% \begin{enumerate}
%     \item Относительная невязка сдвигов (отношение ошибки к оцененному значению) принимает значения от 10.7 \% до 12.3 \%.
%     \item Точность оценки $k, b$ улучшается при $dx, dy \rightarrow 0$.
% \end{enumerate}

% Также были проведены эксперименты с различными коэффициентами линейного преобразования (при фиксированных $(dx, dy)$, Табл. \ref{tab:expon_kb}). Результаты:
% \begin{enumerate}
%     \item Относительная невязка сдвигов $(dx, dy)$ (отношение ошибки к оцененному значению) принимает значения от 10.7 \% до 12.3 \%.
%     \item Оценки сдвигов $(dx, dy)$ от задаваемых величин $(k, b)$ не зависят.
% \end{enumerate}

% % Здесь табличка: Модель / Оценка
% Please add the following required packages to your document preamble:


% \begin{table}[h!]
% \centering
% \caption{Эксперимент 2. Проверка правильности оценки коэффициентов линейного преобразования и зависимости оценки сдвига от них (не зависит)}
% \label{tab:expon_kb}
% \begin{tabular}{|l|l|l|l|l|l|l|l|}
% \hline
% \multicolumn{8}{|l|}{Зафиксировано в модели: $dx = 0.5; dy = 0.25$}                                                \\ \hline
% \multicolumn{2}{|l|}{Модель} & \multicolumn{4}{l|}{Оценка}         & \multicolumn{2}{l|}{Отн. невязка сдвигов}     \\ \hline
% k             & b            & k     & b     & dx, пикс & dy, пикс & $\delta(dx)/dx, \%$ & $\delta(dy)/dy, \%$     \\ \hline
% 1.5           & 0            & 1.47  & 0.007 & 0.57     & 0.28     & 12.3                & 10.7                    \\ \hline
% 5             & 0            & 4.91  & 0.025 & 0.57     & 0.28     & 12.3                & 10.7                    \\ \hline
% 10            & 0            & 9.83  & 0.05  & 0.57     & 0.28     & 12.3                & 10.7                    \\ \hline
% 1             & 0.5          & 0.983 & 0.5   & 0.57     & 0.28     & 12.3                & 10.7                    \\ \hline
% 1             & 5            & 0.983 & 5     & 0.57     & 0.28     & 12.3                & 10.7                    \\ \hline
% 1             & 50           & 0.983 & 50    & 0.57     & 0.28     & 12.3                & 10.7                    \\ \hline
% 1.5           & 0.5          & 1.47  & 0.51  & 0.57     & 0.28     & 12.3                & 10.7                    \\ \hline
% 10            & 50           & 9.83  & 50    & 0.57     & 0.28     & 12.3                & 10.7                    \\ \hline
% \end{tabular}
% \end{table}
%
%%%%%%%%%%%
%

\newpage
\subsection{Эксперимент 3. Оценка значений высоты облаков при различных моделях юстровки}
% Здесь краткое описание эксперимента
% Имеется реальное изображение облачного неба. На нем выделяется фрагмент (200x100) и другой фрагмент (200x100), смещенный на $1, 2 \dots Resize\_factor/2$ пикселей(Рис. \ref{fig:exp3_original}) относительно первого,
% где $Resize\_factor > 1$ --- коэффициент уменьшения ширины и высоты фрагментов. Уменьшаем размеры изображения в $Resize\_factor$ раз (до 50x25 на Рис. \ref{fig:exp3_0.5_-0.25}).

% Далее, значения параметров сдвига $g = (k, b, \delta_x, \delta_y)$ оцениваются при решении задачи \eqref{eq:min_problem}:
% $$
%     \Big\| Ag - \xi \Big\|^2 \rightarrow \min_g .
% $$


\begin{figure}[H]
  \centering
  \includegraphics[width=\linewidth]{images/mag_diss/plot_20160831-154449-796_50.png}
  \caption{Сравнение результатов оценки высоты нижней границы облачности,
  полученных разработанной системой со значением, полученным лазерным
  дальномером (красная линия)}
  \label{fig:laser_cloud_height}
\end{figure}


% Результаты серии экспериментов по оценке субпиксельных сдвигов можно наблюдать в Табл. \ref{tab:subpix_real} (при фиксированных $(k, b)$).

% Можно заметить, что:
% \begin{itemize}
%     \item Относительная невязка сдвигов находится в пределах $[1.4; 31.6] \% $.
%     \item Для перестановок абсолютных значений местами ($dx \leftrightarrow dy$) и разных знаков сдвигов --- вычисленные значения варьируются.
% \end{itemize}


\begin{table}[H]
\centering
\caption{Результаты оценки высоты облаков для различных фрагментов на 3-х стереопарах}
\label{table:height}
\begin{tabular}{|l|l|l|l|l|}
\hline
\textnumero        & Модель A, м   & Модель D3, м & Модель D35, м & Значение дальномера, м \\ \hline
\multirow{3}{*}{1} & $2716 \pm 440$  & $2716 \pm 440$ & $4345 \pm 977$  & \multirow{3}{*}{1600 --- 1700}\\
                   & $2716 \pm 440$  & $3104 \pm 546$ & $3621 \pm 709$  &                               \\
                   & $2897 \pm 488$  & $2716 \pm 440$ & $3104 \pm 546$  &                               \\ \hline
%
\multirow{3}{*}{2} & $2556 \pm 399$  & $2716 \pm 440$ & $4828 \pm 118$2 & \multirow{3}{*}{2300 --- 3200}\\
                   & $3104 \pm 546$  & $3342 \pm 619$ & $4828 \pm 118$2 &                               \\
                   & $2716 \pm 440$  & $2897 \pm 488$ & $4345 \pm 977$  &                               \\ \hline
%
\multirow{6}{*}{3} & $1889 \pm 254$  & $1975 \pm 271$ & $3104 \pm 546$  & \multirow{6}{*}{2200}         \\
                   & $1889 \pm 254$  & $2069 \pm 290$ & $2414 \pm 365$  &                               \\
                   & $1889 \pm 254$  & $1810 \pm 240$ & $2414 \pm 365$  &                               \\
                   & $1889 \pm 254$  & $1889 \pm 254$ & $2172 \pm 312$  &                               \\
                   & $2069 \pm 290$  & $2287 \pm 337$ & $3342 \pm 619$  &                               \\
                   & $1975 \pm 271$  & $2069 \pm 290$ & $2414 \pm 365$  &                               \\ \hline
\end{tabular}
\end{table}
