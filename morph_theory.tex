\subsection{Морфологические методы анализа изображений}\label{morphology}
Для совмещения фрагментов изображений в настоящей работе используются методы морфологического анализа изображений. Рассмотрим базовую теорию из работы \cite{book:morpho}, необходимую для построения модели изображения и применения морфологических методов для его анализа.

\subsubsection{Основные понятия}\label{morph_basic}

Под \emph{изображением} будем понимать числовую функцию $f(\cdot)$, заданную на ограниченном подмножестве $X$ (\emph{поле зрения}) пространства $R^2$. Значения этой функции в точке будем называть \emph{яркостью} в данной точке:
\begin{equation*}
    f(x) = \sum\limits_{i=1}^N c_i\chi_i (x),\quad x \in X ,
\end{equation*}
где \emph{поле зрения} $X$ разбито на области $A_i \subset X$, все точки области $A_i$ имеют одинаковую яркость $c_i$,
\begin{equation*}
    \chi_i(x) =
        \begin{cases}
            1, & x \in A_i\\
            0, & x \notin A_i
        \end{cases}
\end{equation*}
--- индикаторная функция множества $A_i, \; A_i \cap A_j = \varnothing$ при $i \neq j; \; \bigcup_{i=1}^N A_i = X$.


Определим линейные операции сложения изображений и умножения изображения на число следующим образом:
\begin{equation*}
    (f + g)(x) = f(x) + g(x), \quad (a \cdot f)(x) = a \cdot f(x), \quad x \in X, a \in R^1
\end{equation*}
Тогда множество всех изображений, заданных на поле зрения X, представляет собой линейное пространство.


Обозначим $\mu(\cdot)$ некоторую меру на $\sigma$-алгебре борелевских подмножеств поля зрения $X$. Будем считать, что существует интеграл от квадрата яркости по полю зрения $X$:
\begin{equation}\label{eq:intfnotinf}
    \int\limits_X f^2(x)d\mu(x) < \infty
\end{equation}


В качестве меры $\mu$ подмножества $A$ поля зрения $X$ будем использовать либо его
площадь (меру Лебега), либо так называемую считающую меру, когда на поле $X$ задано конечное множество точек (узлов сетки), и мера множества $A \subset X$ равна числу узлов сетки, принадлежащих множеству $A$. Если задана сетка узлов $\{x_1, \dots ,x_n\} \in X$ и считающая мера, то
\begin{equation*}
    \int\limits_Xf(x)d\mu(x) =  \sum\limits_{i=1}^n f(x_i)
\end{equation*}


\begin{samepage}

Если выполнено (\ref{eq:intfnotinf}), то для любых двух изображений $f$ и $g$ можно определить скалярное произведение $(g, f)$, норму $\|f\|$ и расстояние между изображениями $\rho(f, g)$:
\begin{subequations}
\begin{align}
    (g, f)     & = \int\limits_X f(x)g(x)d\mu(x),\label{eq:prostr_scalar} \\
    \|f\|      & = \left(\int\limits_X f^2(x)d\mu(x)\right)^{1/2},\label{eq:prostr_norm} \\
    \rho(f, g) & = \|f - g\| = \left(\int\limits_X (f(x) - g(x))^2d\mu(x)\right)^{1/2}\label{eq:prostr_distance}
\end{align}
\end{subequations}

\end{samepage}


В случае, когда на поле зрения задана сетка узлов, указанные выше формулы можно представить как:
\begin{subequations}
\begin{align*}
    (g, f)     & =  \sum\limits_{i=1}^n f(x_i)g(x_i), \\
    \|f\|      & = \left(\sum\limits_{i=1}^n  f^2(x_i)\right)^{1/2}, \\
    \rho(f, g) & = \|f - g\| = \left(\sum\limits_{i=1}^n (f(x_i) - g(x_i))^2\right)^{1/2}
\end{align*}
\end{subequations}


Определенное таким образом линейное пространство изображений со скалярным произведением (\ref{eq:prostr_scalar}) называется \emph{евклидовым пространством} $\mathcal{L}_{\mu}^2(X)$.


Рассмотрим множество изображений вида:
\begin{equation*}
    g(x) = F(f(x)) \equiv  \sum\limits_{i=1}^{N}F(c_i)\chi(x),
\end{equation*}
где функция $F(\cdot)$ --- любая из некоторого класса $F$ числовых функций, заданных на числовой прямой $R^1$. Далее для таких изображений с преобразованной яркостью будем использовать обозначение $g = F \circ f \in \mathcal{L}$:
\begin{equation*}
g(x) = (F \circ f)(x) \equiv F(f(x)), \quad x \in X.
\end{equation*}

\begin{samepage}

Естественно рассматривать в качестве множества всех возможных преобразований яркости класс $\mathbf{F}_f$ всех таких функций, для которых результирующее изображение $F \circ f, F \in \mathbf{F}_f$, тоже является элементом пространства $\mathcal{L}_{\mu}^2(X)$.
Для этого $F \circ f$ должна быть функцией, определенной на $X$, квадратично интегрируемой --- для этого достаточно, чтобы $F \in \mathbf{F_f}$ были ограниченными борелевскими функциями.

\end{samepage}


\textbf{Определение 1.} Пусть $\mathcal{L}$ --- линейное нормированное пространство всех изображений, $\mathbf{F}$ --- класс всех борелевских функций, определенных  на действительной прямой и принимающих числовые значения, $\mathbf{F}_f$ — подкласс $\mathbf{F}$, выделенный условием
\begin{equation*}
    \mathbf{F}_f = \{F \in \mathbf{F} : \mathbf{F} \circ \mathbf{f}(\cdot) \in \mathcal{L} \}.
\end{equation*}
\begin{enumerate}
\item Будем говорить что \emph{форма изображения $\tilde{f}$ не сложнее, чем форма $f$}, и писать $\tilde{f} \prec f$, если $\tilde{f}(x) = F(f(x)), x \in X$, для некоторой функции $F(\cdot) \in \mathbf{F}_f$.
\item \emph{Формой изображения} $f(\cdot) \in \mathcal{L}$ назовем множество
\begin{equation*}
\mathcal{V}_f = \left\{F \circ f, F \in \mathbf{F}_f \right\} \subset \mathcal{L}
\end{equation*}
\item Изображения $\tilde{f}$ и $f$ назовем \emph{эквивалентными по форме}, если $\tilde{f} \prec f$ и $f \prec \tilde{f}$. Факт эквивалентности изображений будем отмечать как $\tilde{f} \sim f$.
\item Изображения $\tilde{f}$ и $f$ назовем \emph{совпадающими по форме}, если $\mathcal{V}_f = \mathcal{V}_{\tilde{f}}$, в этом случае будем писать $\tilde{f} \equiv f$.
\item Изображения $\tilde{f}$ и $f$ назовем \emph{сравнимыми по форме}, если выполнено либо $f \prec \tilde{f}$, либо $\tilde{f} \prec f$.
\end{enumerate}

Заметим, что $\tilde{f} \equiv f$ влечет $\tilde{f} \sim f$.\\


Согласно этому определению, \emph{форма $\mathcal{V}_f$ изображения $f$ состоит из тех и только из тех изображений $\tilde{f} \in \mathcal{L}$, для которых выполнено $\tilde{f} \prec f$:}
\begin{equation*}
    \mathcal{V}_f = \left\{\tilde{f}: \tilde{f} \prec f\right\};
\end{equation*}
иными словами, \emph{множество $\mathcal{V}_f$ есть множество всех изображений, форма которых не сложнее, чем форма $f$.} Заметим, что все изображения из $\mathcal{V}_f$ сравнимы по форме с $f$, но не обязательно сравнимы по форме между собой.

%
%%%%%%%%%%%%%%%%%%%%%%%%%%%%%%%%%%%%%%%%%%%%%%%%%%%%%%%%%%%%
%

\subsubsection{Форма изображения как оператор проецирования на множество $\mathcal{V}_f$ в пространстве $\mathcal{L}_{\mu}^2(X)$}\label{label}


Чтобы конструктивно воспользоваться понятием формы изображений, заметим, что в случае, если множество всех
изображений есть евклидово пространство $\mathcal{L}_{\mu}^2(X)$, а
множество $\mathcal{V}_f$ является выпуклым замкнутым множеством в $\mathcal{L}_{\mu}^2(X)$, то с каждым подпространством $\mathcal{V}_f \subset \mathcal{L}_{\mu}^2(X)$ взаимно однозначно связан оператор $P_{\mathcal{V}_f}$
ортогонального проецирования на $\mathcal{V}_f$.
Этот оператор каждому элементу $g \in \mathcal{L}_{\mu}^2(X)$ ставит в соответствие его единственную ортогональную проекцию $P_{\mathcal{V}_f}g$ из $\mathcal{V}_f$, определяемую как ближайшее к $g \in \mathcal{L}_{\mu}^2(X)$ изображение $P_{\mathcal{V}_f}g$ из $\mathcal{V}_f$.


Для нахождения проекции следует решить задачу наилучшего приближения элемента $g \in \mathcal{L}_{\mu}^2(X)$
элементами из $\mathcal{V}_f$, т.е. следующую задачу на минимум:
\begin{equation*}
\|g - P_{\mathcal{V}_f}g\|^2 = \inf_{h}\left\{\|g-h\|^2\quad|\quad h \in \mathcal{V}_f\right\}.
\end{equation*}
Проекция $P_{\mathcal{V}_f}g$ изображения $g$ на форму $\mathcal{V}_f$ является изображением из множества $\mathcal{V}_f$, наиболее близким к $g$.


Множество $\mathcal{V}_f$ можно записать как множество собственных элементов оператора ортогонального проецирования $P_{\mathcal{V}_f}$:
\begin{equation}\label{eq:vfdef}
\mathcal{V}_f = \left\{g \in \mathcal{L}_{\mu}^2(X): P_{\mathcal{V}_f}g = g \right\}.
\end{equation}

Поскольку оператор ортогонального проецирования в ряде случаев легко вычисляется, то вместо множества $\mathcal{V}_f$
можно использовать взаимно однозначно связанный с ним проектор $P_{\mathcal{V}_f}$. Этот оператор ортогонального проецирования
в морфологическом анализе тоже называется \emph{формой изображения} $f$.

\begin{samepage}

В случае, если $\mathcal{V}_f$ является $N$-мерным линейным подпространством и задано соотношением:
\begin{equation*}
\mathcal{V}_f = \left\{  f(x) = \sum\limits_{i=1}^{N} c_i(x)\chi_i(x), \quad x \in X,\quad c_i \in (-\infty, \infty),\quad i=1, \dots ,N \right\},
\end{equation*}
то
\begin{equation}\label{eq:pvfg_sum}
P_{\mathcal{V}_f}g = \sum\limits_{i=1}^{N}\frac{(g, \chi_i)}{\|\chi_i\|^2}\chi_i
\end{equation}
Записав в явном виде входящие в (\ref{eq:pvfg_sum}) функции получим
(учтем, что $\chi^2_i(x) \equiv \chi_i(x)$)
\begin{equation}\label{eq:pvfg_sum_full}
P_{\mathcal{V}_f}g(x) = \sum\limits_{i=1}^{N}\frac{ \int\limits_Xg(x')\chi_i(x')d\mu(x') }{ \int\limits_X\chi_i(x')d\mu(x') }\chi_i(x), \quad x \in X
\end{equation}
Это соотношение выполнено для всех точек множества $X$, кроме точек нулевой меры.

\end{samepage}


Соотношение (\ref{eq:pvfg_sum}) означает, что проекция $g \in \mathcal{L}_{\mu}^2(X)$ на $\mathcal{V}_f$ есть мозаичное изображение $P_{\mathcal{V}_f}g$ с множествами постоянной
яркости, совпадающими с множествами постоянной яркости изображений из класса $\mathcal{V}_f$. Яркость изображения на
каждом множестве $A_i$ равна
средней яркости изображения $g$ на множестве $A_i, \; i=1, \dots ,N$.


Явный вид ортогонального проектора (\ref{eq:pvfg_sum_full}) на подпространство $\mathcal{V}_f$ показывает, что морфологические методы со строго определенным математически понятием формы изображения в виде (\ref{eq:vfdef}) являются легко реализуемыми как на обычных цифровых компьютерах, так и на спецпроцессорах.

%
%%%%%%%%%%%%%%%%%%%%%%%%%%%%%%%%%%%%%%%%%%%%%%%%%%%%%%%%%%%%
%


\subsubsection{Метод поиска фрагмента изображения заданной формы}\label{morph_search}
Пусть задано изображение $f(\cdot) \in \mathcal{L}_{\mu}^2(X)$, его форма есть замкнутое выпуклое множество
\begin{equation*}
\mathcal{V}_f = \left\{ h(\cdot) \in \mathcal{L}_{\mu}^2(X) : h(\cdot) = F \circ f(\cdot), \quad F \in \mathbf{F}_f \right\},
\end{equation*}
где $\mathbf{F}_f$ --- класс всех борелевских функций, таких, что $F \circ f \in \mathcal{L}_{\mu}^2(X)$ для всех $F \in \mathbf{F}_f$, и обозначим $P_f \in \left( \mathcal{L}_{\mu}^2(X) \rightarrow \mathcal{L}_{\mu}^2(X) \right) $ оператор ортогонального проецирования в $\mathcal{L}_{\mu}^2(X)$ на это множество. Обозначим $P_0$ оператор ортогонального проецирования на множество

\begin{equation*}
V = \left\{ c \cdot \chi_{H}(\cdot), c \in R^1 \right\} \subset \mathcal{L}_{\mu}^2(X),
\end{equation*}
где $H$ --- некоторая область изображения (множество точек).
%где индекс $\gamma$ означает всевозможные деформации (сдвиги, повороты) области изображения $H$ (у нас будут только сдвиги).


Тогда для любого изображения $q(\cdot) \in \mathcal{L}_{\mu}^2(X)$ значение функционала
\begin{equation}\label{eq:tau}
\tau(q, f) = \tau(F \circ q, f) = \frac{\|q - P_fq\|^2}{\|P_0q - P_fq\|^2}, \quad F \in \mathbf{F}_f
\end{equation}
согласно его определению, будет тем меньше, чем больше отличие $P_fq$ от константы, и чем меньше отличие $q$ от $f$ по форме.

Таким образом, значение функционала (\ref{eq:tau}) может быть использовано как критерий морфологический близости двух изображений.


Зададим подмножество $H$ поля зрения $X$, указав те точки, которые входят в заданное подмножество. Это можно сделать, задав индикаторную функцию следующим образом:

\begin{equation*}
\chi_H(x) =
\begin{cases}
   1, & x \in H\\
   0, & x \notin H
\end{cases}
\end{equation*}


Пусть эталонное изображение, заданное на $H$, имеет вид кусочно-постоянного изображения
\begin{equation*}
h_x(x) = \sum\limits_{i=1}^{N}c_i\chi_i(x),\quad x \in H.
\end{equation*}
Здесь $\chi_i$ --- индикаторная функция множества $A_i \in H, i = 1, \dots , N; \;$ множества $A_1, \dots ,A_N$ образуют разбиение множества $H$ на непересекающиеся подмножества. Форма $\mathcal{V}_h$ эталонного изображения $h$ задается как множество всех изображений такого вида.


Поиск фрагмента изображения, схожего по форме с эталоном, осуществляется передвижением множества $\chi_H$ по всему полю зрения и вычислением меры близости фрагмента изображения $f$, накрываемого подвижным множеством $H$, с эталоном формы.
Проекция фрагмента изображения на область $H$ поля зрения $X$, сдвинутую на вектор $z_0$ от начала отсчета, определяется аналогично  \eqref{eq:pvfg_sum_full} и вычисляется  по  вычисляется по формуле
$$
\sum \limits_{i=1}^{N} \frac{\int \limits_X f(x)\chi_i(x-z_0)d\mu(x)}{\int \limits_X \chi_i(x)d\mu(x)}.
$$
Для этого приходится вычислять интегралы вида:
\begin{equation} \label{eq:convolution}
    \int\limits_H f(z-z_0)\chi_A(z)dz,
\end{equation}
где $A$ --- одно из подмножеств $H$, а $z_0$ --- вектор сдвига подмножества $H$ по полю зрения.


Заметим, что интегралы необходимые для вычисления близости (\ref{eq:tau}) имеют вид, похожий на свертку функция $f$ и $\chi_A$. Чтобы они действительно стали сверткой, вычисляемой по всему полю зрения, надо функции $\chi_A(\cdot)$ продолжить нулем на все поле зрения $X$ и интегрировать по $X$. Все функции, заданные на $X$, следует продолжить периодически на всю плоскость $R^2$. Формулу (\ref{eq:tau}) можно представить как:

\begin{samepage}

\begin{equation}
    \tau(f, h) = \frac{\|f - P_hf\|^2}{\|P_0f - P_hf\|^2} =
    \frac{ I_{f^2}(z_0) - \sum\limits_{i=1}^{N}\frac{I_i^2(z_0)}{k(A_i)} }
    { \sum\limits_{i=1}^{N}\frac{I_i^2(z_0)}{k(A_i)} - \|P_0f\|^2 },
\end{equation}
где
\begin{subequations}
\begin{align*}
    I_i(z_0)     & = \int\limits_X f(z-z_0)\chi_{A_i}(z)dz, \\
    I_0(z_0)     & = \sum_{i=1}^{N}I_i(z_0), \\
    I_{f^2}(z_0) & = \int\limits_X f^2(z-z_0)\chi_X(z)dz,
\end{align*}
\end{subequations}
а $k(A_i)$ --- число точек в множестве $A_i$.

\end{samepage}


Действительно,

\begin{equation*}
    P_hf(z_0) = \sum_{i=1}^{N} \frac{I_i(z_0)}{k(A_i)} \chi_i(x)
\end{equation*}
--- проекция $f$ на кусочно-постоянный эталон $h$.

\begin{equation*}
    P_0f(z_0) = \frac{I_0}{\textup{число точек поля зрения}} \chi_H
\end{equation*}
--- проекция $f$ на область $H$ с постоянной (и равной средней по всему изображению) яркостью.


Вычислим теперь квадраты норм $\|f - P_hf\|_{z_0}^2$ и $\|P_0f - P_hf\|_{z_0}^2$ для области изображения $H$, заданной вектором $z_0$, для каждого фиксированного значения $z_0$. Раскрыв скобки, заметим, что
\begin{equation*} %||f - P_hf||
\begin{split}
    \|f - P_hf\|_{z_0}^2
    &= \|f\|_{z_0}^2 -2(f, P_hf)_{z_0} + \|P_hf\|_{z_0}^2 \\
    &= \Bigg| \; \textup{Заметим,} \quad (f, P_hf) = (P_hf, P_hf) = \|P_hf\|^2 \; \Bigg|  \\
    &= \|f\|^2_{z_0} - \|P_hf\|^2_{z_0}
    = \underbrace{\|f\|^2}_{const} - \sum\limits_{i=1}^{N}\frac{I_i^2(z_0)}{k(A_i)} ,
\end{split}
\end{equation*}

Заметим, $P_h$ действует(усредняет) только в <<текущем окошке>> $H$, а $P_0$ усредняет по всему изображению (полю зрения) $X$. Поэтому, действуем мы $P_h$ на $f$ или нет --- все равно, т.к. потом в любом случае подействует $P_0$ и мы получим везде среднюю по всему изображению яркость.
\begin{equation}
\begin{split}
    \|P_hf - P_0f\|_{z_0}^2
	&= \Bigg|  \;
    P_0 = P_0P_h = P_hP_0 \; \Bigg|
    = \|P_hf\|_{z_0}^2 -2\underbrace{(P_hf, P_0f)_{z_0}}_{= \|P_0f\|^2} + \|P_0f\|_{z_0}^2 \\
    &= \|P_hf\|^2_{z_0} - \|P_0f\|^2_{z_0}
    = \sum\limits_{i=1}^{N}\frac{I_i^2(z_0)}{k(A_i)} - \underbrace{\|P_0f\|^2}_{const} , \\
     \textup{где} \;
    \|P_0f\|^2_{z_0} &\equiv \|P_0f\|^2
    = \frac{I_0^2}{ \textup{число точек поля зрения} } \; \textup{--- не зависит от $z_0$.}
\end{split}
\end{equation}


Операция свертки может быть вычислена с использованием дискретного преобразования Фурье\cite{book:fastfure}. Дискретное преобразование Фурье можно вычислять с помощью алгоритма быстрого преобразования Фурье, который имеет вычислительную сложность $O(n \cdot log(n))$. Именно этот алгоритм используется в работе для вычисления функционала близости при поиске фрагмента изображения.

%
%%%%%%%%%%%%%%%%%%%%%%%%%%%%%%%%%%%%%%%%%%%%%%%%%%%%%%%%%%%%
%
