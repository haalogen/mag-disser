\begin{center}
  \section*{Введение}
  \addcontentsline{toc}{section}{Введение}
\end{center}

Облачность оказывает существенное воздействие на земную климатическую систему,
участвуя в гидрологическом цикле и влияя на радиационный баланс земной атмосферы.
Характеристики облачности необходимо учитывать при дистанционном зондирования
атмосферы и поверхности Земли наземными и космическими средствами. Для оценки этих
характеристик облачности необходимо иметь простую и надежную систему,
позволяющую получать достоверную информацию об облачности.

В фотограмметрии есть подраздел стереоскопии, изучающий использование стереокамер (оптических систем с несколькими окулярами).
С помощью такой стереосистемы можно оценить расстояние до объекта по принципу <<бинокулярного зрения>>.
В частности, такой метод используется в
ИФА им. А.М.~Обухова РАН для определения высоты нижней границы облачности \cite{art:refined}---\cite{art:estimation4}.
Этот метод был разработан в дипломной работе М.С.Андреева \cite{book:andreev_diplom}.
Он обладает преимуществом перед стандартными лазерными дальномерами --- он значительно более экономичен
(необходимы две камеры с одинаковыми характеристиками и программное обеспечение, обрабатывающее получаемые пары изображений --- стереопары).
Данный метод дает удовлетворительную точность при определении высоты, однако наблюдается следующая зависимость: чем выше облачность,
тем меньше дистанция между соответствующими точками на изображении, и тем больше относительная погрешность оценки высоты (доходит до 100 \%).
Настоящая дипломная работа посвящена устранению недостатков, проявившихся при применении метода, разработанного в \cite{book:andreev_diplom}.
Для решения этой проблемы, в первую очередь, было решено использовать изображения большего разрешения (3072x2304 вместо 1600x1200).
При этом мы сталкиваемся с проблемой возрастания числа вычислений и дисторсионными искажениями изображений.
Для уменьшения числа вычислений было решено использовать аппроксимацию кусочно-постоянным изображением с меньшим числом градаций
(деление спектра: 256 $\rightarrow$ 32 цветов, например) и учет априорных сведений о характерных высотах облаков (200 м~---~10 км в Москве).
Коэффициенты дисторсионного искажения оптической системы можно оценить, а затем несложным способом избавиться от влияния дисторсии.

Следующим этапом решения задачи уменьшения погрешности высоты может быть применение модели субпиксельного совмещения изображений.
Повышение точности определения сдвигов кадров стереопары также приведет к желаемому уменьшению погрешности высоты, но в данной работе
этот подход не рассматривается.

Целью данной работы является разработка математического и программного обеспечения информационной системы для оценки высоты и скорости
движения облаков по их изображениям высокого разрешения.

Для достижения обозначенных выше целей в работе решаются следующие задачи:
\begin{itemize}
  \item[--] Уточнение модели юстировки фотоаппаратов. Учет дисторсии.
  \item[--] Выбор параметров модели дисторсии на основе параметра надежности методами теории измерительно-вычислительных систем.
  \item[--] Уточнение методов и алгоритмов морфологического совмещения фрагментов изображения облачности.
  \item[--] Анализ точности решения задачи определения расстояния до облаков и скорости их движения.
\end{itemize}





