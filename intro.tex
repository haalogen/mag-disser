\begin{center}
  \section*{Введение}
  \addcontentsline{toc}{section}{Введение}
\end{center}

Задача совмещения изображений, снятых одновременно двумя камерами, оптические оси которых параллельны, имеет очевидное прикладное значение. С помощью такой системы можно оценивать расстояние до объекта (бинокулярное зрение). В частности, такой метод используется в ИФА им. А.М.~Обухова РАН для определения высоты нижней границы облачности \cite{art:refined}---\cite{art:estimation4}. Этот метод был разработан в дипломной работе М.С.Андреева \cite{book:andreev_diplom}.
 Он обладает преимуществом перед стандартными лазерными дальномерами --- он значительно более экономичен (необходимы две камеры с одинаковыми характеристиками и программное обеспечение, обрабатывающее получаемые пары изображений --- стереопары). Данный метод дает удовлетворительную точность при определении высоты, однако наблюдается следующая зависимость: чем выше облачность, тем меньше дистанция между соответствующими точками на изображении, и тем больше относительная погрешность оценки высоты (доходит до 100 \%). Настоящая дипломная работа посвящена устранению недостатков, проявившихся при применении метода, разработанного в \cite{book:andreev_diplom}.
 Для решения этой проблемы, в первую очередь, было решено использовать изображения большего разрешения (3072x2304 вместо 1600x1200). При этом мы сталкиваемся с проблемой возрастания числа вычислений и дисторсионными искажениями изображений. Для уменьшения числа вычислений было решено использовать аппроксимацию кусочно-постоянным изображением с меньшим числом градаций (деление спектра: 256 $\rightarrow$ 32 цветов, например) и учет априорных сведений о характерных высотах облаков (200 м~---~10 км в Москве). Коэффициенты дисторсионного искажения оптической системы можно оценить, а затем несложным способом избавиться от влияния дисторсии.

Следующим этапом решения задачи уменьшения погрешности высоты является разработка и применение модели субпиксельного совмещения изображений. Повышение точности определения сдвигов кадров стереопары также приведет к желаемому уменьшению погрешности высоты.

Целью данной работы является разработка и экспериментальная проверка методов совмещения изображений с субпиксельной точностью, а также методов избавления от дисторсий камер на больших изображениях.

Для достижения вышеобозначенных целей в работе решаются следующие задачи:
построение модели аффинной связи между кадрами с учетом влияния дисторсий и экспериментальная проверка этой модели;
построение модели субпиксельного совмещения двух кадров стереопары
и экспериментальная проверка модели субпиксельного совмещения;
построение модели кусочно-постоянной аппроксимации изображений с целью уменьшения числа вычислений (метод деления спектра)
и экспериментальная проверка метода деления спектра.

