\documentclass[a4paper,14pt]{extarticle}

                        % Поддержка русского языка
\usepackage{cmap}       % Поддержка поиска русских слов в PDF (pdflatex)

\usepackage{indentfirst}% Красная строка в первом абзаце

\usepackage{mathtext}   % если нужны русские буквы в формулах
                        % перенос слов с тире
\lccode`\-=`\-
\defaulthyphenchar=127
            % Выбор языка и кодировки
\usepackage[T2A]{fontenc}
\usepackage[utf8]{inputenc}
\usepackage[english, russian]{babel}

\usepackage[hidelinks]{hyperref}
\usepackage{graphicx}
\graphicspath{{./images/}}
\usepackage{float}
\usepackage{amsmath, amsthm, amssymb}
\usepackage{subcaption}
\usepackage{multirow}

\linespread{1.3} %эквивалентно полуторному интервалу
\usepackage{setspace}
\onehalfspacing

\usepackage{chngcntr}
\counterwithin{figure}{section}
\counterwithin{equation}{section}
\counterwithin{table}{section}

\usepackage[a4paper, left=3cm, right=2cm, top=2cm, bottom=2.5cm]{geometry}

\newcommand{\eps}{\varepsilon}
\newcommand{\distsqr}{(x-x_0)^2 + (y-y_0)^2}
\newcommand{\wtil}{\widetilde}
\begin{document}

\begin{titlepage}
\begin{center}
  ФЕДЕРАЛЬНОЕ ГОСУДАРСТВЕННОЕ БЮДЖЕТНОЕ ОБРАЗОВАТЕЛЬНОЕ УЧРЕЖДЕНИЕ ВЫСШЕГО ОБРАЗОВАНИЯ «МОСКОВСКИЙ ГОСУДАРСТВЕННЫЙ УНИВЕРСИТЕТ\\
  имени М.В. ЛОМОНОСОВА» \\
  ФИЗИЧЕСКИЙ ФАКУЛЬТЕТ\\
  КАФЕДРА МАТЕМАТИЧЕСКОГО МОДЕЛИРОВАНИЯ И ИНФОРМАТИКИ
\end{center}

\vfill

\begin{center}
  МАГИСТЕРСКАЯ ДИССЕРТАЦИЯ\\
  {\bf <<МАТЕМАТИЧЕСКОЕ МОДЕЛИРОВАНИЕ СИСТЕМЫ АНАЛИЗА ИЗОБРАЖЕНИЙ ОБЛАКОВ ДЛЯ ОЦЕНКИ ИХ ВЫСОТЫ И СКОРОСТИ ДВИЖЕНИЯ>>}
\end{center}

\vfill

\begin{flushright}
  Выполнил студент\\
  235М группы:\\
  Никитин Станислав Викторович\\
  \underline{\hspace{4cm}} \hspace{0.3cm} \! \\
  {\footnotesize подпись студента }\vspace{1cm}\\
  Научный руководитель: \\
  профессор Чуличков Алексей Иванович\\
  \underline{\hspace{4cm}} \hspace{0.3cm} \! \\
  {\footnotesize подпись научного руководителя }
\end{flushright}


\begin{flushleft}
Допущен к защите <<\underline{\hspace{0.8cm}}>> \underline{\hspace{1.1cm}} 2018\\
Зав. кафедрой \underline{\hspace{4cm}}\\
\hspace{4cm} {\footnotesize подпись зав. кафедрой}
\end{flushleft}

\vfill

\begin{center}
  Москва \\
  2018
\end{center}
\end{titlepage}

\newpage
\setcounter{page}{2}
\tableofcontents

\begin{center}
  \section*{Введение}
  \addcontentsline{toc}{section}{Введение}
\end{center}

Задача совмещения изображений, снятых одновременно двумя камерами, оптические оси которых параллельны, имеет очевидное прикладное значение. С помощью такой системы можно оценивать расстояние до объекта (бинокулярное зрение). В частности, такой метод используется в ИФА им. А.М.~Обухова РАН для определения высоты нижней границы облачности \cite{art:refined}---\cite{art:estimation4}. Этот метод был разработан в дипломной работе М.С.Андреева \cite{book:andreev_diplom}.
 Он обладает преимуществом перед стандартными лазерными дальномерами --- он значительно более экономичен (необходимы две камеры с одинаковыми характеристиками и программное обеспечение, обрабатывающее получаемые пары изображений --- стереопары). Данный метод дает удовлетворительную точность при определении высоты, однако наблюдается следующая зависимость: чем выше облачность, тем меньше дистанция между соответствующими точками на изображении, и тем больше относительная погрешность оценки высоты (доходит до 100 \%). Настоящая дипломная работа посвящена устранению недостатков, проявившихся при применении метода, разработанного в \cite{book:andreev_diplom}.
 Для решения этой проблемы, в первую очередь, было решено использовать изображения большего разрешения (3072x2304 вместо 1600x1200). При этом мы сталкиваемся с проблемой возрастания числа вычислений и дисторсионными искажениями изображений. Для уменьшения числа вычислений было решено использовать аппроксимацию кусочно-постоянным изображением с меньшим числом градаций (деление спектра: 256 $\rightarrow$ 32 цветов, например) и учет априорных сведений о характерных высотах облаков (200 м~---~10 км в Москве). Коэффициенты дисторсионного искажения оптической системы можно оценить, а затем несложным способом избавиться от влияния дисторсии.

Следующим этапом решения задачи уменьшения погрешности высоты является разработка и применение модели субпиксельного совмещения изображений. Повышение точности определения сдвигов кадров стереопары также приведет к желаемому уменьшению погрешности высоты.

Целью данной работы является разработка и экспериментальная проверка методов совмещения изображений с субпиксельной точностью, а также методов избавления от дисторсий камер на больших изображениях.

Для достижения вышеобозначенных целей в работе решаются следующие задачи:
построение модели аффинной связи между кадрами с учетом влияния дисторсий и экспериментальная проверка этой модели;
построение модели субпиксельного совмещения двух кадров стереопары
и экспериментальная проверка модели субпиксельного совмещения;
построение модели кусочно-постоянной аппроксимации изображений с целью уменьшения числа вычислений (метод деления спектра)
и экспериментальная проверка метода деления спектра.




\begin{center}\section{Математические модели  и методы анализа  изображений}\end{center}
\subsection{Геометрическая модель эксперимента}\label{geometry}

\subsubsection{Математическая модель стереопары изображений}
Для оценки расстояния до объекта по стереопаре его изображений опишем сначала математическую модель стереопары изображений при условии, что обе фотокамеры имеют одни и те же параметры (увеличение) и технические характеристики, а их оптические оси направлены в зенит. Согласно \cite{book:andreev_diplom} и \cite{article:stereo}, мы можем использовать следующую модель.

\begin{figure}
\center{\includegraphics[width=1\linewidth]{scheme}}
\caption{Схема формирования изображения точки облака двумя фотокамерами}
\label{img:scheme}
\end{figure}


С каждой фотокамерой свяжем декартову систему координат, в которой ось $O_iZ$ направлена вертикально вверх и совпадает с ее оптической осью, горизонтальная ось $O_iX$ проходит через центры камеры и направлена от камеры $S_L$ к камере $S_R, \; i=1,2$. Пусть облако расположено на высоте $D$, расстояние между фотокамерами по горизонтальной оси равно $B$ (размер \emph{базы}). Пусть $x_0$ - горизонтальный размер пространственной области на высоте $D$, изображаемой фотокамерой, $\varphi_0$ - угол обзора камеры (Рис.~\ref{img:scheme}). Тогда
\begin{equation}\label{eq:D_first}
    D = \frac{x_0}{2tg(\frac{\varphi_0}{2})}
\end{equation}


Выберем точку облака (для наглядности на Рис.~\ref{img:scheme} эта точка лежит в плоскости, проходящей через оптические оси обеих камер). Обозначим $x_1$ и $x_2$ координаты выбранной точки облака вдоль осей $O_1X$ и $O_2X$ в системе координат первой и второй камер соответственно. Для точки облака, расположенной в плоскости оптических осей, выполнено равенство  $|x_1 - x_2| = B$. Поэтому вместо \eqref{eq:D_first} имеем:
\begin{equation}\label{eq:D_second}
    D = \frac{Bx_0}{2tg(\frac{\varphi0}{2})|x_1 - x_2|}
\end{equation}


В формуле (\ref{eq:D_second}) наблюдаемыми являются параметры $B$ (стереобаза), $\varphi_0$ (угол обзора фотокамеры) и отношение $\frac{x_0}{|x_1-x_2|}$ - оно равно отношению $\frac{\xi_0}{|\xi_1 - \xi_2|}$ размера кадра $\xi_0$ к модулю разности координат изображений
выбранной точки облака на первом и втором кадре $|\xi_1 - \xi_2|$. Итак, формула для вычисления высоты облака есть
\begin{equation}\label{eq:D_final}
  D = \frac{B\xi_0}{2tg\frac{\varphi_0}{2}|\xi_1 - \xi_2|}
\end{equation}

%
%%%%%%%%%%%%%%%%%%%%%%%%%%%%%%%%%%%%%%%%%%%%%%%%%%%%%%%%%%%%%%%%%%
%

\subsubsection{Оценка погрешности определения расстояния до облака}

Погрешность определения высоты облака с помощью формулы (\ref{eq:D_final}) оценим по формуле для погрешности косвенных измерений:
\begin{equation*}
    \Delta D =  \sqrt{ (\frac{\partial D}{\partial B}\Delta B)^2 + (\frac{\partial D}{\partial \xi_0}\Delta \xi_0)^2 + (\frac{\partial D}{\partial \varphi_0}\Delta \varphi_0)^2 + (\frac{\partial D}{\partial \xi_1}\Delta \xi_1)^2 + (\frac{\partial D}{\partial \xi_2}\Delta \xi_2)^2 }
\end{equation*}
Заметим, что $\Delta \xi_0 = 0$, т.к. размеры изображения мы всегда знаем точно. Также, обычно $\Delta \xi_1 = \Delta \xi_2$, поэтому формула упрощается:
\begin{equation*}
    \Delta D =  \sqrt{ (\frac{\partial D}{\partial B}\Delta B)^2 + (\frac{\partial D}{\partial \varphi_0}\Delta \varphi_0)^2 + 2(\frac{\partial D}{\partial \xi_1}\Delta \xi_1)^2}
\end{equation*}

Если проанализировать, какие слагаемые дают наибольший вклад в погрешность, то заметим, что подавляющую часть вносят погрешности  $(\frac{\partial D}{\partial \xi_i}\Delta \xi_i), \; i=1,2.$  Существенно уменьшить эти погрешности можно:
\begin{itemize}
\item увеличивая размер изображений;
\item повышая точность определения координат $\xi_i$ (<<субпиксельная точность>>).
\end{itemize}
И то и другое приведет к росту числа вычислений. Поэтому целесообразно разработать и применить методы к их сокращению:
\begin{itemize}
\item деление спектра Grayscale-изображения (яркость $0 \dots 255$) на $m < 256$ промежутков, соответствующих новым (средним по промежутку) яркостям;
\item ограничение области поиска фрагмента, ближайшего к эталону, из соображений характерных высот облаков (от 200 м до 10 км).
\end{itemize}

%
%%%%%%%%%%%%%%%%%%%%%%%%%%%%%%%%%%%%%%%%%%%%%%%%%%%%%%%%%%%%%%%%%%
%

\subsubsection{Оценка скорости движения облаков}
Рассмотрим два снимка облака сделанные одной и той же камерой с разницей во
времени $\Delta t$. Пусть перемещение фрагмента облака в плоскости соответствует
сдвигам $\Delta x, \Delta y$ (в метрах) и пиксельным сдвигам
$\Delta \xi, \Delta \eta$ изображений. Введем обозначения (см. \ref{img:cloud_velocity_scheme}):
\begin{align*}
  R &= \sqrt{\Delta x^2 + \Delta y^2}\\
  r &= \sqrt{\Delta \xi^2 + \Delta \eta^2}
\end{align*}

\begin{figure}[H]
\centering
\includegraphics[height=0.4\textheight]{images/mag_diss/cloud_velocity_scheme.png}
\caption{Схема геометричекой модели измерения скорости движения объекта}
\label{img:cloud_velocity_scheme}
\end{figure}

Считаем, что высота облака $D$ за время $\Delta t$ изменилась не существенно.
Тогда, из планиметрических соображений можно записать отношение:
\begin{equation}
  \frac{D}{f} = \frac{
    \sqrt{\Delta x^2 + \Delta y^2}
  }{
    \sqrt{\Delta \xi^2 + \Delta \eta^2}
  } = \frac{R}{r}
\end{equation}
Здесь $f$ --- фокусное расстояние камеры (порядка 50 мм), $D$ --- высота
облачности.

Тогда скорость облака можем оценить как
\begin{equation}
  v = \frac{R}{\Delta t} =  \frac{D \cdot r}{\Delta t \cdot f}
\end{equation}

Значения скорости различных облаков могут варьироваться в пределах от 10 до
200 м/с.
\subsection{Юстировка камер}\label{cam_tuning}
В описанной в предыдущем разделе математической модели одним из требований является параллельность оптических осей обеих камер и одинаковый угол обзора.
К этому добавляется условие, что направления координатных осей $O_1X$ и $O_2X$, связанные с кадрами первой и второй камер, должны совпадать. В реальности выполнить эти требования довольно сложно. Однако, если иметь тестовые изображения одного и того же объекта, то можно преобразовать одно из изображений облака к виду, который оно имело бы при выполнении вышеназванных условий.

В настоящей работе, так же, как и в работе \cite{book:andreev_diplom}, в качестве тестовых объектов использовалось звездное небо. Звезды можно считать бесконечно удаленными точечными объектами, поэтому их изображения на фотокамерах, оси координат которых совмещены в соответствии с указанными выше требованиями, должны быть полностью идентичными, то есть координаты изображений одной и той же звезды на левой и правой камерах должны совпадать. Если этого не происходит, то камеры не юстированы должным образом.

Вместо совмещения осей координат поворотами камер найдем такое преобразование координат второй камеры, при котором изображения бесконечно удаленных объектов будут совпадать.

Однако при обработке изображений большого разрешения (3072x2304 в нашем случае) существенны искажения, связанные с дисторсиями оптических систем. Поэтому, избавимся в первую очередь от дисторсионных искажений.
%
%%%%%%%%%%%%%%%%%%%%%%%%%%%%%%%%%%%%%%%%%%%%%%
%

\subsubsection{Учет дисторсий фотокамер и аффинные преобразования}\label{distortion_affine}

\begin{figure}[h]
\center{ \includegraphics[width=1\linewidth]{stars_with_lines_600x400_mod.png} }
\caption{Прямые, проходящие через центр кадра, содержат изображения звезд первого и второго кадра. Это свидетельствует о наличии дисторсии.}
\label{img:starlines}
\end{figure}

При увеличении размера кадра до 3072х2304 пикселей начинает сказываться отличие используемой  модели геометрической оптики от реальности. Действительно, на рисунке \ref{img:starlines} представлены два кадра стереопары, наложенные друг на друга после аффинного преобразования; на одном из них звезды обозначены синим цветом, на другом – красным. Видно, что координаты звезд не совпадают, но если провести прямые через центр кадра и изображения звезд на первом и втором кадрах, получим систему прямых, к каждой из которых близки изображения одной и той же звезды каждого кадра. Это свидетельствует о наличии дисторсии \cite[стр.~73]{book:optic}. Учтем этот факт и уточним математическую модель связи между координатами звезд на правом и левом кадрах.

Пусть $(\xi_1, \eta_1)$ --- координаты точки объекта (звезды) на первом изображении, $(\xi_2, \eta_2)$ --- ее координаты на втором изображении.
Из-за дисторсии эти координаты отличаются от \emph{идеальных} координат $(x_1, y_1)$ и $(x_2, y_2)$, формируемых идеальными оптическими системами, причем таким образом, что изображение точки смещается к центру (либо от центра) кадра на величину, пропорциональную $l^3(x, y)$, где $l(x, y)$ --- расстояние от $(x, y)$ до центра кадра $(x_0, y_0)$ \cite[стр.~73]{book:optic}.

\begin{figure}[h]
\center{ \includegraphics[width=0.8\linewidth]{triangle_dist} }
\caption{Связь координат $(x, y)$ идеальной системы и координат $(\xi, \eta)$ системы с дисторсиями. Точка $B(x, y)$, точка $B'(\xi, \eta)$.}
\label{img:triangle_dist}
\end{figure}


Найдем связь между координатами $(x, y)$ идеального изображения звезды и ее координатами $(\xi, \eta)$ на реальном снимке. Для этого рассмотрим
два треугольника, $OBC$ и $OB'C'$, на рисунке \ref{img:triangle_dist}. Координаты точки В есть $(x, y)$, а точки $B'$ --- $(\xi, \eta)$, $O$ ---  центр кадра с координатами $(x_0, y_0)$.

Эти треугольники подобны, поэтому $ \frac{OC'}{OC} = \frac{OB'}{OB} $,
подставляя координаты точек $B, B', C $ и $ C'$, учитывая, что
$$
    |BB'| = \eps l^3 = \eps \left( \sqrt{ \distsqr } \right)^3 ,
$$
имеем
$$
    \frac{ (x-x_0) + (\xi-x) }{ x-x_0 } =
    \frac{ \Big(\distsqr \Big)^{1/2} + \eps \Big((x-x_0)^2 + (y-y_0)^2 \Big)^{3/2} }
    { \Big(\distsqr \Big)^{1/2} },
$$
откуда
$$
    \xi - x = \eps (x-x_0) \Big( \distsqr \Big), \\
$$
Аналогично, из равенства $ \frac{B'C'}{BC} = \frac{OB'}{OB} $
$$
    \eta - y = \eps (y-y_0) \Big( \distsqr \Big).
$$

Таким образом, искомая связь имеет вид
\begin{equation} \label{distort_transform}
\begin{split}
    x_i &= \xi_i - \eps_i z_x(x_i, y_i), \\
    y_i &= \eta_i - \eps_i z_y(x_i, y_i), \; i = 1,2 \;, \;
\textup{где} \\
    z_x(x, y) &= (x-x_0) \Big( \distsqr \Big), \\
    z_y(x, y) &= (y-y_0) \Big( \distsqr \Big),
\end{split}
\end{equation}
$(x_0, y_0)$ --- координаты центра кадра, а $\eps_1$ и $\eps_2$ --- коэффициенты дисторсий первого и второго кадра соответственно.

Координаты $(x_1, y_1)$ и $(x_2, y_2)$ звезд на первом и втором кадрах, сформированные идеальными видеокамерами, связаны аффинным преобразованием
\begin{equation} \label{pure_affine}
    \left( \begin{array}{c} x_2 \\ y_2 \end{array} \right) =
    \begin{pmatrix} a & b \\ c & d \end{pmatrix}
    \left( \begin{array}{c} x_1 \\ y_1 \end{array} \right) +
    \left( \begin{array}{c} e \\ f \end{array} \right)
\end{equation}

Подставив \eqref{distort_transform} в \eqref{pure_affine}, получим
\begin{equation*}
    \left( \begin{array}{c} \xi_2 - \eps_2 z_x(x_2, y_2) \\
    \eta_2 - \eps_2 z_y(x_2, y_2) \end{array} \right) =
    \begin{pmatrix} a & b \\ c & d \end{pmatrix}
    \left( \begin{array}{c} \xi_1 - \eps_1 z_x(x_1, y_1) \\
    \eta_1 - \eps_1 z_y(x_1, y_1) \end{array} \right) +
    \left( \begin{array}{c} e \\ f \end{array} \right)
\end{equation*}

Считая коэффициенты дисторсий $\eps_1$ и $\eps_2$ малыми по сравнению с единицей, коэффициенты $a$ и $d$ аффинных преобразований близкими к единице, $b$ и $c$ --- малыми по сравнению с единицей, и заменяя $z_x(x_i, y_i), \; z_y(x_i, y_i)$ на близкие к ним значения $z_x(\xi_i, \eta_i), \; z_y(\xi_i, \eta_i)$, запишем линеаризованную связь между наблюдаемыми величинами и неизвестными коэффициентами $g = (a, b, c, d, e, f, \eps_1, \eps_2)$:
\begin{equation}
  \begin{split}
  \left( \begin{array}{c} \xi_2^{(1)} \\ \eta_2^{(1)} \\ \vdots \\
  \xi_2^{(N)} \\ \eta_2^{(N)} \end{array} \right)
  %
  & = \begin{pmatrix} \xi_1^{(1)} & \eta_1^{(1)} & 0 & 0 & 1 & 0 &
  -z_x(\xi_1^{(1)}, \eta_1^{(1)}) & z_x(\xi_2^{(1)}, \eta_2^{(1)})\\
  0 & 0 & \xi_1^{(1)} & \eta_1^{(1)} & 0 & 1 &
  -z_y(\xi_1^{(1)}, \eta_1^{(1)}) & z_y(\xi_2^{(1)}, \eta_2^{(1)})\\
  %
  \vdots & \vdots & \vdots & \vdots & \vdots & \vdots & \vdots & \vdots \\
  %
  \xi_1^{(N)} & \eta_1^{(N)} & 0 & 0 & 1 & 0 &
  -z_x(\xi_1^{(N)}, \eta_1^{(N)}) & z_x(\xi_2^{(N)}, \eta_2^{(N)})\\
  0 & 0 & \xi_1^{(N)} & \eta_1^{(N)} & 0 & 1 &
  -z_y(\xi_1^{(N)}, \eta_1^{(N)}) & z_y(\xi_2^{(N)}, \eta_2^{(N)})\\
  \end{pmatrix}\\
  %
  & \times \begin{pmatrix} a & b & c & d & e & f & \eps_1 & \eps_2 \end{pmatrix}^T
  \end{split}
\end{equation}
где $N$ – число точечных объектов (звезд) на изображении.

Или в векторном виде, с учетом погрешности измерений,
\begin{equation} \label{scheme_measure}
    \xi = A g + \nu,
\end{equation}
где $g \in R^8$ --- вектор параметров преобразования, матрица $A$ размера $2N \times 8$ связывает вектор параметров $g$ с координатами точечных объектов на первом кадре. Вектор $\nu \in R^{2N}$ --- вектор погрешностей измерения координат точечных объектов. Матрицу $A$ считаем известной точно, вектор $g$ произволен, вектор $\nu$ --- случайный с нулевым математическим ожиданием и матрицей ковариаций $\Sigma$. В настоящей работе считалось, что координаты вектора $\nu$ некоррелированы и дисперсия координат равнялась квадрату удвоенного расстояния между соседними  пикселями цифрового изображения, так, что $\Sigma = \sigma^2 I$, где $I$ --- единичная матрица, $\sigma^2 = 4 \; (\textup{пиксель})^2$ (берем с запасом, чтобы покрыть также возможные погрешности в матрице $A$).

Как было показано в \cite[стр.~141]{book:pytev_ivs} и \cite{book:exper}, решением задачи редукции для данной модели является следующая оценка $\hat{f}$:
\begin{equation*}
\hat{f} = R\xi,
R = (A^* \Sigma^{-1} A)^{-1} A^* \Sigma^{-1} = A^- ,
\end{equation*}
т.к. $\Sigma$ --- диагональная матрица, $A^*$ --- транспонированная.

Из измерений \eqref{scheme_measure} оцениваются коэффициенты $g = (a, b, c, d, e, f, \eps_1, \eps_2)$, определяющие связь координат звезд на двух кадрах, далее юстировка осуществляется пересчетом изображений к виду, какой они имели бы, если бы дисторсии отсутствовали, а оптические оси камер и оси координат кадров первой и второй камер совпадали.

Пересчет координат производится по следующим формулам.
\begin{itemize}
    \item Пусть $f_1$ --- изображение на первом кадре, тогда при отсутствии дисторсий изображение того же объекта будет равно
\begin{equation} \label{eq:f1}
    \widetilde{f_1}(x, y) = f_1 \Big( x+\eps_1 z_x(x, y),\; y+\eps_1 z_y(x, y) \Big)
\end{equation}

    \item Пусть $f_2$ --- изображение на втором кадре, тогда при отсутствии дисторсий и рассогласований в направлениях оптической оси и осей координат видеокамер изображение того же объекта на втором кадре будет равно
\begin{equation} \label{eq:f2}
  \begin{split}
    & \wtil{f_2}(x, y) = f_2 \Big( \wtil{x}(x, y),\; \wtil{y}(x, y) \Big), \\
    %
    & \textup{где} \;
    %
    \left( \begin{array}{c} \wtil{x} \\ \wtil{y} \end{array} \right) =
    \begin{pmatrix} a & b \\ c & d \end{pmatrix}
    %
    \left( \begin{array}{c}
    \underbrace{x+\eps_2 z_x(x, y)}_{= \xi_2} \\ \underbrace{y+\eps_2 z_y(x, y)}_{= \eta_2}
    \end{array} \right) +
    \left( \begin{array}{c} e \\ f \end{array} \right)
  \end{split}
\end{equation}

\end{itemize}

Яркости пикселей изображений $\wtil{f_1}$ и $\wtil{f_2}$ получаются из \eqref{eq:f1}---\eqref{eq:f2} как значения функций $\wtil{f_1}(\cdot, \cdot)$ и $\wtil{f_2}(\cdot, \cdot)$  в узлах равномерной прямоугольной
сетки и вычисляются из яркостей пикселей изображений $f_1$ и $f_2$ с использованием интерполяции.


\subsubsection{Модель с учетом дисторсий 3го и 5го порядков}
Используя рассуждения, аналогичные предыдущей части, можно получить связь координат, учитывающую дисторсионные искажения 3го и 5го порядков. В этом случае решением задачи является вектор параметров $g = (a, b, c, d, e, f, \eps_1, \eps_2, \mu_1, \mu_2) \in R^{10}$.


\subsection{Морфологические методы анализа изображений}\label{morphology}
Для совмещения фрагментов изображений в настоящей работе используются методы морфологического анализа изображений. Рассмотрим базовую теорию из работы \cite{book:morpho}, необходимую для построения модели изображения и применения морфологических методов для его анализа.

\subsubsection{Основные понятия}\label{morph_basic}

Под \emph{изображением} будем понимать числовую функцию $f(\cdot)$, заданную на ограниченном подмножестве $X$ (\emph{поле зрения}) пространства $R^2$. Значения этой функции в точке будем называть \emph{яркостью} в данной точке:
\begin{equation*}
    f(x) = \sum\limits_{i=1}^N c_i\chi_i (x),\quad x \in X ,
\end{equation*}
где \emph{поле зрения} $X$ разбито на области $A_i \subset X$, все точки области $A_i$ имеют одинаковую яркость $c_i$,
\begin{equation*}
    \chi_i(x) =
        \begin{cases}
            1, & x \in A_i\\
            0, & x \notin A_i
        \end{cases}
\end{equation*}
--- индикаторная функция множества $A_i, \; A_i \cap A_j = \varnothing$ при $i \neq j; \; \bigcup_{i=1}^N A_i = X$.


Определим линейные операции сложения изображений и умножения изображения на число следующим образом:
\begin{equation*}
    (f + g)(x) = f(x) + g(x), \quad (a \cdot f)(x) = a \cdot f(x), \quad x \in X, a \in R^1
\end{equation*}
Тогда множество всех изображений, заданных на поле зрения X, представляет собой линейное пространство.


Обозначим $\mu(\cdot)$ некоторую меру на $\sigma$-алгебре борелевских подмножеств поля зрения $X$. Будем считать, что существует интеграл от квадрата яркости по полю зрения $X$:
\begin{equation}\label{eq:intfnotinf}
    \int\limits_X f^2(x)d\mu(x) < \infty
\end{equation}


В качестве меры $\mu$ подмножества $A$ поля зрения $X$ будем использовать либо его
площадь (меру Лебега), либо так называемую считающую меру, когда на поле $X$ задано конечное множество точек (узлов сетки), и мера множества $A \subset X$ равна числу узлов сетки, принадлежащих множеству $A$. Если задана сетка узлов $\{x_1, \dots ,x_n\} \in X$ и считающая мера, то
\begin{equation*}
    \int\limits_Xf(x)d\mu(x) =  \sum\limits_{i=1}^n f(x_i)
\end{equation*}


\begin{samepage}

Если выполнено (\ref{eq:intfnotinf}), то для любых двух изображений $f$ и $g$ можно определить скалярное произведение $(g, f)$, норму $\|f\|$ и расстояние между изображениями $\rho(f, g)$:
\begin{subequations}
\begin{align}
    (g, f)     & = \int\limits_X f(x)g(x)d\mu(x),\label{eq:prostr_scalar} \\
    \|f\|      & = \left(\int\limits_X f^2(x)d\mu(x)\right)^{1/2},\label{eq:prostr_norm} \\
    \rho(f, g) & = \|f - g\| = \left(\int\limits_X (f(x) - g(x))^2d\mu(x)\right)^{1/2}\label{eq:prostr_distance}
\end{align}
\end{subequations}

\end{samepage}


В случае, когда на поле зрения задана сетка узлов, указанные выше формулы можно представить как:
\begin{subequations}
\begin{align*}
    (g, f)     & =  \sum\limits_{i=1}^n f(x_i)g(x_i), \\
    \|f\|      & = \left(\sum\limits_{i=1}^n  f^2(x_i)\right)^{1/2}, \\
    \rho(f, g) & = \|f - g\| = \left(\sum\limits_{i=1}^n (f(x_i) - g(x_i))^2\right)^{1/2}
\end{align*}
\end{subequations}


Определенное таким образом линейное пространство изображений со скалярным произведением (\ref{eq:prostr_scalar}) называется \emph{евклидовым пространством} $\mathcal{L}_{\mu}^2(X)$.


Рассмотрим множество изображений вида:
\begin{equation*}
    g(x) = F(f(x)) \equiv  \sum\limits_{i=1}^{N}F(c_i)\chi(x),
\end{equation*}
где функция $F(\cdot)$ --- любая из некоторого класса $F$ числовых функций, заданных на числовой прямой $R^1$. Далее для таких изображений с преобразованной яркостью будем использовать обозначение $g = F \circ f \in \mathcal{L}$:
\begin{equation*}
g(x) = (F \circ f)(x) \equiv F(f(x)), \quad x \in X.
\end{equation*}

\begin{samepage}

Естественно рассматривать в качестве множества всех возможных преобразований яркости класс $\mathbf{F}_f$ всех таких функций, для которых результирующее изображение $F \circ f, F \in \mathbf{F}_f$, тоже является элементом пространства $\mathcal{L}_{\mu}^2(X)$.
Для этого $F \circ f$ должна быть функцией, определенной на $X$, квадратично интегрируемой --- для этого достаточно, чтобы $F \in \mathbf{F_f}$ были ограниченными борелевскими функциями.

\end{samepage}


\textbf{Определение 1.} Пусть $\mathcal{L}$ --- линейное нормированное пространство всех изображений, $\mathbf{F}$ --- класс всех борелевских функций, определенных  на действительной прямой и принимающих числовые значения, $\mathbf{F}_f$ — подкласс $\mathbf{F}$, выделенный условием
\begin{equation*}
    \mathbf{F}_f = \{F \in \mathbf{F} : \mathbf{F} \circ \mathbf{f}(\cdot) \in \mathcal{L} \}.
\end{equation*}
\begin{enumerate}
\item Будем говорить что \emph{форма изображения $\tilde{f}$ не сложнее, чем форма $f$}, и писать $\tilde{f} \prec f$, если $\tilde{f}(x) = F(f(x)), x \in X$, для некоторой функции $F(\cdot) \in \mathbf{F}_f$.
\item \emph{Формой изображения} $f(\cdot) \in \mathcal{L}$ назовем множество
\begin{equation*}
\mathcal{V}_f = \left\{F \circ f, F \in \mathbf{F}_f \right\} \subset \mathcal{L}
\end{equation*}
\item Изображения $\tilde{f}$ и $f$ назовем \emph{эквивалентными по форме}, если $\tilde{f} \prec f$ и $f \prec \tilde{f}$. Факт эквивалентности изображений будем отмечать как $\tilde{f} \sim f$.
\item Изображения $\tilde{f}$ и $f$ назовем \emph{совпадающими по форме}, если $\mathcal{V}_f = \mathcal{V}_{\tilde{f}}$, в этом случае будем писать $\tilde{f} \equiv f$.
\item Изображения $\tilde{f}$ и $f$ назовем \emph{сравнимыми по форме}, если выполнено либо $f \prec \tilde{f}$, либо $\tilde{f} \prec f$.
\end{enumerate}

Заметим, что $\tilde{f} \equiv f$ влечет $\tilde{f} \sim f$.\\


Согласно этому определению, \emph{форма $\mathcal{V}_f$ изображения $f$ состоит из тех и только из тех изображений $\tilde{f} \in \mathcal{L}$, для которых выполнено $\tilde{f} \prec f$:}
\begin{equation*}
    \mathcal{V}_f = \left\{\tilde{f}: \tilde{f} \prec f\right\};
\end{equation*}
иными словами, \emph{множество $\mathcal{V}_f$ есть множество всех изображений, форма которых не сложнее, чем форма $f$.} Заметим, что все изображения из $\mathcal{V}_f$ сравнимы по форме с $f$, но не обязательно сравнимы по форме между собой.

%
%%%%%%%%%%%%%%%%%%%%%%%%%%%%%%%%%%%%%%%%%%%%%%%%%%%%%%%%%%%%
%

\subsubsection{Форма изображения как оператор проецирования на множество $\mathcal{V}_f$ в пространстве $\mathcal{L}_{\mu}^2(X)$}\label{label}


Чтобы конструктивно воспользоваться понятием формы изображений, заметим, что в случае, если множество всех
изображений есть евклидово пространство $\mathcal{L}_{\mu}^2(X)$, а
множество $\mathcal{V}_f$ является выпуклым замкнутым множеством в $\mathcal{L}_{\mu}^2(X)$, то с каждым подпространством $\mathcal{V}_f \subset \mathcal{L}_{\mu}^2(X)$ взаимно однозначно связан оператор $P_{\mathcal{V}_f}$
ортогонального проецирования на $\mathcal{V}_f$.
Этот оператор каждому элементу $g \in \mathcal{L}_{\mu}^2(X)$ ставит в соответствие его единственную ортогональную проекцию $P_{\mathcal{V}_f}g$ из $\mathcal{V}_f$, определяемую как ближайшее к $g \in \mathcal{L}_{\mu}^2(X)$ изображение $P_{\mathcal{V}_f}g$ из $\mathcal{V}_f$.


Для нахождения проекции следует решить задачу наилучшего приближения элемента $g \in \mathcal{L}_{\mu}^2(X)$
элементами из $\mathcal{V}_f$, т.е. следующую задачу на минимум:
\begin{equation*}
\|g - P_{\mathcal{V}_f}g\|^2 = \inf_{h}\left\{\|g-h\|^2\quad|\quad h \in \mathcal{V}_f\right\}.
\end{equation*}
Проекция $P_{\mathcal{V}_f}g$ изображения $g$ на форму $\mathcal{V}_f$ является изображением из множества $\mathcal{V}_f$, наиболее близким к $g$.


Множество $\mathcal{V}_f$ можно записать как множество собственных элементов оператора ортогонального проецирования $P_{\mathcal{V}_f}$:
\begin{equation}\label{eq:vfdef}
\mathcal{V}_f = \left\{g \in \mathcal{L}_{\mu}^2(X): P_{\mathcal{V}_f}g = g \right\}.
\end{equation}

Поскольку оператор ортогонального проецирования в ряде случаев легко вычисляется, то вместо множества $\mathcal{V}_f$
можно использовать взаимно однозначно связанный с ним проектор $P_{\mathcal{V}_f}$. Этот оператор ортогонального проецирования
в морфологическом анализе тоже называется \emph{формой изображения} $f$.

\begin{samepage}

В случае, если $\mathcal{V}_f$ является $N$-мерным линейным подпространством и задано соотношением:
\begin{equation*}
\mathcal{V}_f = \left\{  f(x) = \sum\limits_{i=1}^{N} c_i(x)\chi_i(x), \quad x \in X,\quad c_i \in (-\infty, \infty),\quad i=1, \dots ,N \right\},
\end{equation*}
то
\begin{equation}\label{eq:pvfg_sum}
P_{\mathcal{V}_f}g = \sum\limits_{i=1}^{N}\frac{(g, \chi_i)}{\|\chi_i\|^2}\chi_i
\end{equation}
Записав в явном виде входящие в (\ref{eq:pvfg_sum}) функции получим
(учтем, что $\chi^2_i(x) \equiv \chi_i(x)$)
\begin{equation}\label{eq:pvfg_sum_full}
P_{\mathcal{V}_f}g(x) = \sum\limits_{i=1}^{N}\frac{ \int\limits_Xg(x')\chi_i(x')d\mu(x') }{ \int\limits_X\chi_i(x')d\mu(x') }\chi_i(x), \quad x \in X
\end{equation}
Это соотношение выполнено для всех точек множества $X$, кроме точек нулевой меры.

\end{samepage}


Соотношение (\ref{eq:pvfg_sum}) означает, что проекция $g \in \mathcal{L}_{\mu}^2(X)$ на $\mathcal{V}_f$ есть мозаичное изображение $P_{\mathcal{V}_f}g$ с множествами постоянной
яркости, совпадающими с множествами постоянной яркости изображений из класса $\mathcal{V}_f$. Яркость изображения на
каждом множестве $A_i$ равна
средней яркости изображения $g$ на множестве $A_i, \; i=1, \dots ,N$.


Явный вид ортогонального проектора (\ref{eq:pvfg_sum_full}) на подпространство $\mathcal{V}_f$ показывает, что морфологические методы со строго определенным математически понятием формы изображения в виде (\ref{eq:vfdef}) являются легко реализуемыми как на обычных цифровых компьютерах, так и на спецпроцессорах.

%
%%%%%%%%%%%%%%%%%%%%%%%%%%%%%%%%%%%%%%%%%%%%%%%%%%%%%%%%%%%%
%


\subsubsection{Метод поиска фрагмента изображения заданной формы}\label{morph_search}
Пусть задано изображение $f(\cdot) \in \mathcal{L}_{\mu}^2(X)$, его форма есть замкнутое выпуклое множество
\begin{equation*}
\mathcal{V}_f = \left\{ h(\cdot) \in \mathcal{L}_{\mu}^2(X) : h(\cdot) = F \circ f(\cdot), \quad F \in \mathbf{F}_f \right\},
\end{equation*}
где $\mathbf{F}_f$ --- класс всех борелевских функций, таких, что $F \circ f \in \mathcal{L}_{\mu}^2(X)$ для всех $F \in \mathbf{F}_f$, и обозначим $P_f \in \left( \mathcal{L}_{\mu}^2(X) \rightarrow \mathcal{L}_{\mu}^2(X) \right) $ оператор ортогонального проецирования в $\mathcal{L}_{\mu}^2(X)$ на это множество. Обозначим $P_0$ оператор ортогонального проецирования на множество

\begin{equation*}
V = \left\{ c \cdot \chi_{H}(\cdot), c \in R^1 \right\} \subset \mathcal{L}_{\mu}^2(X),
\end{equation*}
где $H$ --- некоторая область изображения (множество точек).
%где индекс $\gamma$ означает всевозможные деформации (сдвиги, повороты) области изображения $H$ (у нас будут только сдвиги).


Тогда для любого изображения $q(\cdot) \in \mathcal{L}_{\mu}^2(X)$ значение функционала
\begin{equation}\label{eq:tau}
\tau(q, f) = \tau(F \circ q, f) = \frac{\|q - P_fq\|^2}{\|P_0q - P_fq\|^2}, \quad F \in \mathbf{F}_f
\end{equation}
согласно его определению, будет тем меньше, чем больше отличие $P_fq$ от константы, и чем меньше отличие $q$ от $f$ по форме.

Таким образом, значение функционала (\ref{eq:tau}) может быть использовано как критерий морфологический близости двух изображений.


Зададим подмножество $H$ поля зрения $X$, указав те точки, которые входят в заданное подмножество. Это можно сделать, задав индикаторную функцию следующим образом:

\begin{equation*}
\chi_H(x) =
\begin{cases}
   1, & x \in H\\
   0, & x \notin H
\end{cases}
\end{equation*}


Пусть эталонное изображение, заданное на $H$, имеет вид кусочно-постоянного изображения
\begin{equation*}
h_x(x) = \sum\limits_{i=1}^{N}c_i\chi_i(x),\quad x \in H.
\end{equation*}
Здесь $\chi_i$ --- индикаторная функция множества $A_i \in H, i = 1, \dots , N; \;$ множества $A_1, \dots ,A_N$ образуют разбиение множества $H$ на непересекающиеся подмножества. Форма $\mathcal{V}_h$ эталонного изображения $h$ задается как множество всех изображений такого вида.


Поиск фрагмента изображения, схожего по форме с эталоном, осуществляется передвижением множества $\chi_H$ по всему полю зрения и вычислением меры близости фрагмента изображения $f$, накрываемого подвижным множеством $H$, с эталоном формы.
Проекция фрагмента изображения на область $H$ поля зрения $X$, сдвинутую на вектор $z_0$ от начала отсчета, определяется аналогично  \eqref{eq:pvfg_sum_full} и вычисляется  по  вычисляется по формуле
$$
\sum \limits_{i=1}^{N} \frac{\int \limits_X f(x)\chi_i(x-z_0)d\mu(x)}{\int \limits_X \chi_i(x)d\mu(x)}.
$$
Для этого приходится вычислять интегралы вида:
\begin{equation} \label{eq:convolution}
    \int\limits_H f(z-z_0)\chi_A(z)dz,
\end{equation}
где $A$ --- одно из подмножеств $H$, а $z_0$ --- вектор сдвига подмножества $H$ по полю зрения.


Заметим, что интегралы необходимые для вычисления близости (\ref{eq:tau}) имеют вид, похожий на свертку функция $f$ и $\chi_A$. Чтобы они действительно стали сверткой, вычисляемой по всему полю зрения, надо функции $\chi_A(\cdot)$ продолжить нулем на все поле зрения $X$ и интегрировать по $X$. Все функции, заданные на $X$, следует продолжить периодически на всю плоскость $R^2$. Формулу (\ref{eq:tau}) можно представить как:

\begin{samepage}

\begin{equation}
    \tau(f, h) = \frac{\|f - P_hf\|^2}{\|P_0f - P_hf\|^2} =
    \frac{ I_{f^2}(z_0) - \sum\limits_{i=1}^{N}\frac{I_i^2(z_0)}{k(A_i)} }
    { \sum\limits_{i=1}^{N}\frac{I_i^2(z_0)}{k(A_i)} - \|P_0f\|^2 },
\end{equation}
где
\begin{subequations}
\begin{align*}
    I_i(z_0)     & = \int\limits_X f(z-z_0)\chi_{A_i}(z)dz, \\
    I_0(z_0)     & = \sum_{i=1}^{N}I_i(z_0), \\
    I_{f^2}(z_0) & = \int\limits_X f^2(z-z_0)\chi_X(z)dz,
\end{align*}
\end{subequations}
а $k(A_i)$ --- число точек в множестве $A_i$.

\end{samepage}


Действительно,

\begin{equation*}
    P_hf(z_0) = \sum_{i=1}^{N} \frac{I_i(z_0)}{k(A_i)} \chi_i(x)
\end{equation*}
--- проекция $f$ на кусочно-постоянный эталон $h$.

\begin{equation*}
    P_0f(z_0) = \frac{I_0}{\textup{число точек поля зрения}} \chi_H
\end{equation*}
--- проекция $f$ на область $H$ с постоянной (и равной средней по всему изображению) яркостью.


Вычислим теперь квадраты норм $\|f - P_hf\|_{z_0}^2$ и $\|P_0f - P_hf\|_{z_0}^2$ для области изображения $H$, заданной вектором $z_0$, для каждого фиксированного значения $z_0$. Раскрыв скобки, заметим, что
\begin{equation*} %||f - P_hf||
\begin{split}
    \|f - P_hf\|_{z_0}^2
    &= \|f\|_{z_0}^2 -2(f, P_hf)_{z_0} + \|P_hf\|_{z_0}^2 \\
    &= \Bigg| \; \textup{Заметим,} \quad (f, P_hf) = (P_hf, P_hf) = \|P_hf\|^2 \; \Bigg|  \\
    &= \|f\|^2_{z_0} - \|P_hf\|^2_{z_0}
    = \underbrace{\|f\|^2}_{const} - \sum\limits_{i=1}^{N}\frac{I_i^2(z_0)}{k(A_i)} ,
\end{split}
\end{equation*}

Заметим, $P_h$ действует(усредняет) только в <<текущем окошке>> $H$, а $P_0$ усредняет по всему изображению (полю зрения) $X$. Поэтому, действуем мы $P_h$ на $f$ или нет --- все равно, т.к. потом в любом случае подействует $P_0$ и мы получим везде среднюю по всему изображению яркость.
\begin{equation}
\begin{split}
    \|P_hf - P_0f\|_{z_0}^2
	&= \Bigg|  \;
    P_0 = P_0P_h = P_hP_0 \; \Bigg|
    = \|P_hf\|_{z_0}^2 -2\underbrace{(P_hf, P_0f)_{z_0}}_{= \|P_0f\|^2} + \|P_0f\|_{z_0}^2 \\
    &= \|P_hf\|^2_{z_0} - \|P_0f\|^2_{z_0}
    = \sum\limits_{i=1}^{N}\frac{I_i^2(z_0)}{k(A_i)} - \underbrace{\|P_0f\|^2}_{const} , \\
     \textup{где} \;
    \|P_0f\|^2_{z_0} &\equiv \|P_0f\|^2
    = \frac{I_0^2}{ \textup{число точек поля зрения} } \; \textup{--- не зависит от $z_0$.}
\end{split}
\end{equation}


Операция свертки может быть вычислена с использованием дискретного преобразования Фурье\cite{book:fastfure}. Дискретное преобразование Фурье можно вычислять с помощью алгоритма быстрого преобразования Фурье, который имеет вычислительную сложность $O(n \cdot log(n))$. Именно этот алгоритм используется в работе для вычисления функционала близости при поиске фрагмента изображения.

%
%%%%%%%%%%%%%%%%%%%%%%%%%%%%%%%%%%%%%%%%%%%%%%%%%%%%%%%%%%%%
%

%\subsection{Модель субпиксельного совмещения изображений}\label{subpixel}
Совмещения изображений с точностью до пикселя порой бывает недостаточно: чем больше дальность (высота) до облака, тем меньше расстояние (в пикселях) между найденными фрагментами совмещаемых изображений. Погрешность такой оценки может доходить до (и даже превышать) 100 \% от вычисленного значения дальности. Поэтому, актуальной является задача дальнейшего увеличения точности совмещения кадров стереопары.

Итак, необходимо построить модель двух кадров, смещенных друг относительно друга на крайне малое расстояние (менее 1 пикселя по каждой координате).

Будем считать, что функция яркости одного($\xi$) и другого($\eta$) кадров связаны малым сдвигом аругментов и линейным преобразованием. Также разложим функцию яркости первого изображения $f(x_i+\delta_x, y_j+\delta_y)$ в ряд Тейлора до линейных членов в точке $(x_i, y_j)$:
\begin{subequations} 
\begin{align}
    \xi_{ij} &= f(x_i+\delta_x, y_j+\delta_y) \approx f(x_i, y_j) 
    + f'_x(x_i, y_j)\delta_x + f'_y(x_i, y_j)\delta_y \label{eq:img_left} \\
    \eta_{ij} &= kf(x_i, y_j) + b , \label{eq:img_right}
\end{align}
\end{subequations}
где $i$ --- абсцисса пикселя, $j$ --- ордината пикселя,
$\delta_x$ --- сдвиг по $x$, $\delta_y$ --- сдвиг по $y$.

Тогда из \eqref{eq:img_left}
\begin{equation}
    f(x_i, y_j) = \xi_{ij} - f'_x(x_i, y_j)\delta_x - 
    f'_y(x_i, y_j)\delta_y,
\end{equation}
где $f'_x(x_i, y_j), \; f'_y(x_i, y_j)$ --- частные производные $f(\cdot, \cdot)$ по $x$ и $y$ соответственно (в точках $x_i, y_j$).

Из \eqref{eq:img_right} получим 
\begin{equation}
    f(x, y) = \frac{\eta_{ij}-b}{k} = \wtil{k}\eta_{ij} + \wtil{b},
\end{equation}
где $\wtil{k}=\frac{1}{k}, \; \wtil{b}=-\frac{b}{k}$.


Необходимо найти $(k, \; b, \; \delta_x, \; \delta_y)$ --- параметры преобразования одного изображения к другому. Для этого необходимо решить следующую задачу на минимум
\begin{equation}
    \sum_{i, j} \Big( \xi_{ij} - f'_x(x_i, y_j)\delta_x - f'_y(x_i, y_j)\delta_y - \wtil{k}\eta_{ij} - \wtil{b} \Big)^2 \rightarrow \min_{g}
\end{equation}
В последней формуле минимизируемый функционал является квадратичной
формой относительно неизвестных коэффициентов.

Оценим $f'_x(x_i, y_j)$ и $f'_y(x_i, y_j)$ из значений изображения $f(x_i+\delta_x, y_j+\delta_y)$, так как не знаем знаков $\delta_x$ и $\delta_y$. При работе с изображениями удобно индексировать точки, начиная с левого верхнего угла (точка $(0,0)$ --- в левом верхнем углу изображения). Мы будем поступать также:
\begin{equation*}
\begin{split}
    \textup{В неграничных точках:} \\
    f'_x(x_i, y_j) &= \frac{\xi_{i+1,j} - \xi_{i-1, j}}{2} \\
    f'_y(x_i, y_j) &= \frac{\xi_{i, j+1} - \xi_{i, j-1}}{2} \\
    \textup{В граничных точках:} \\
    f'_x(x_i, y_j) &= \xi_{i+1,j} - \xi_{i, j} \;\textup{(левая граница)} \\
    f'_x(x_i, y_j) &= \xi_{i,j} - \xi_{i-1, j} \;\textup{(правая граница)} \\
    f'_y(x_i, y_j) &= \xi_{i, j+1} - \xi_{i, j} \;\textup{(верхняя граница)} \\
    f'_y(x_i, y_j) &= \xi_{i, j} - \xi_{i, j-1} \;\textup{(нижняя граница)}
\end{split}
\end{equation*}
Эти формулы дают порядки аппроксимации по шагам равномерной сетки не ниже $O(h_x)$ и $O(h_y)$ \cite{book:chisl_met}.

Перепишем задачу на минимум в векторном виде:
\begin{equation} \label{eq:min_problem}
    \Big\| Ag - \xi \Big\|^2 \rightarrow \min_g
\end{equation}
Пусть $N$ --- число точек по $x$, $M$ --- число точек по $y$. Тогда $\xi = (\xi_{1,1} \dots \xi_{N,M}) \in R^{NM}$ --- вектор всех значений яркости первого изображения. Матрица $A \in R^{NM \times 4}$ заполнена следующим образом:
\begin{equation*}
\begin{split}
    A_{s, 1} &= \eta_{ij} \\
    A_{s, 2} &= 1 \\
    A_{s, 3} &= f'_x(x_i, y_j) \\
    A_{s, 4} &= f'_y(x_i, y_j) \\
    s &= i + jN = 1 \dots NM
\end{split}
\end{equation*}

Решением задачи \eqref{eq:min_problem} является вектор
\begin{equation*}
    g = \left( \begin{array}{c} \wtil{k} \\ \wtil{b} \\ \delta_x \\ \delta_y \end{array} \right) = A^- \xi = \Big( A^* A \Big)^{-1} A^* \;\xi
\end{equation*}

%
%%%%%%%%%%%
%

%\subsection{Деление спектра изображения}
При решении задачи совмещения кадров облаков удобно переходить от цветного (RGB) изображения к Grayscale-изображению (обычно 256 оттенков серого; <<0>> --- черный, <<255>> --- белый цвет). Тогда для каждой из 256 яркостей определена область данной яркости, задающая форму изображения облака. Для каждой из таких областей, в соответствии с предложенным методом и реализующим его вычислительным алгоритмом, следует вычислять интеграл свертки \eqref{eq:convolution}.  Однако, в получаемых изображениях облаков значительная часть Grayscale-спектра (обычно та, что ближе к <<0>>-черному цвету) отсутствует или очень слабо представлена. Поэтому, можно аппроксимировать исходное изображения с 256 уровнями яркости изображением с существенно меньшим числом уровней --- например, с 32 уровнями. Такую операцию будем называть \emph{делением спектра (яркости) изображения}.

Рассмотрим три метода такого деления спектра изображения на интервалы, соответствующие новым цветам (оттенкам):
\begin{enumerate}
    \item Равномерное деление --- самый простой для реализации и самый наивный метод деления, который, однако, дает адекватные результаты (например, при $256 \rightarrow 32$ градации).
    % здесь рисуночек спектра, поделенного равномерно
    
    \item Интегральный метод --- в основе лежит идея о делении спектра на участки, интегральные суммы которых примерно равны: 
$$
    I_1 \approx I_2 \approx \dots \approx I_n \approx \frac{I}{n}.
$$
Сложность алгоритма --- $O(N)$, где $N = 256$ оригинальных градаций.
    % здесь рисуночек спектра, поделенного интегрально
    
    \item Итерационный метод \cite[стр. 132]{book:morpho} --- итеративное деление на новые интервалы; нулевое приближение --- равномерное деление:
$$
    \wtil{c_j} = \textup{средняя яркость области $A_j$ на $m$-той итерации} =
    \underbrace{ \frac{1}{S_j} }_{ \textup{число пикселей} } \sum_{x_k \in A_j} f(x_k)
$$
Новые грaницы областей $A_1 \dots A_n$ на $m$-той итерации:
$$
    \underbrace{ 0 \quad;\quad \frac{\wtil{c}_1+\wtil{c}_2}{2} }_{A_1} \quad\dots\quad 
    \underbrace{ \frac{\wtil{c}_{n-1}+\wtil{c}_n} {2} \quad;\quad 255 }_{A_n}
$$
\end{enumerate}

В качестве критерия остановки итераций можно следить за изменением невязки между нашей кусочно-постоянной аппроксимацией и изначальным изображением $f(\cdot)$:
$$
    \sum_{x \in X} \Big( f(x) - \sum_{i} \wtil{c}_i \chi_i(x) \Big)^2 \rightarrow \min_{\{A_j\}}
$$

Результаты работы трех алгоритмов можно сравнить на Рис. \ref{fig:hist_divider}.

\begin{figure}[H]
\begin{subfigure}{\linewidth}
  \centering
  \includegraphics[height=0.265\textheight]{images/hist_divider/8col_iter0.png}
  \caption{Равномерное деление}
  \label{fig:uniform_div}
\end{subfigure}
\\
\begin{subfigure}{\linewidth}
  \centering
  \includegraphics[height=0.3\textheight]{images/hist_divider/8cols_integral.png}
  \caption{Интегральный метод}
  \label{fig:integral_div}
\end{subfigure}
\\
\begin{subfigure}{\linewidth}
  \centering
  \includegraphics[height=0.265\textheight]{images/hist_divider/8col_iter20.png}
  \caption{Итерационный метод}
  \label{fig:iterative_div}
\end{subfigure}
\caption{Деление спектра изображения облака тремя методами}
\label{fig:hist_divider}
\end{figure}

\begin{center}\section{Описание экспериментов}\end{center}
\subsection{Эксперимент 1. Проверка моделей юстировки камер: аффинной (A), с учетом дисторсии 3 порядка (D3), с учетом дисторсии 3 порядка (D35)}
% Здесь краткое описание эксперимента
В ходе эксперимента вычислялись параметры линеаризованной связи координат левого и правого кадра стереопары. Для каждой из трех моделей виртуальной юстировки решалась задача теории измерительно-вычислительных систем \eqref{scheme_measure} $\xi = A g + \nu$. Ее решение:
$$
    g = (a, b, c, d, e, f, \eps_1, \eps_2, \mu_1, \mu_2) = A^- \xi,
$$
где $A^-$ --- псевдообратная матрица (считается с помощью встроенной функции пакета NumPy numpy.linalg.pinv).

Результаты вычисления параметров связи для всех трех моделей можно видеть в таблице \ref{table:coeffs_reliability}.
Также, в последнем столбце указано значение надежности($\tau$) для найденных коэффициентов.

% Далее, в первую очередь избавляемся от дисторсий на левом и правом изображениях (переписываем старые точки в новые места):
% \begin{python}
% nleft[round(i - eps1 * zx1 - mu1 * wx1), round(j - eps1 * zy1 - mu1 * wy1)] = left[i, j]
% nright[round(i - eps2 * zx2 - mu2 * wx2), round(j - eps2 * zy2 - mu2 * wy2)] = right[i, j]
% \end{python}

% Затем, применяем аффинное преобразование с найденными коэффициентами
% $(a, b, c, d, e, f)$, используя бикубическую интерполяцию ( функция PIL.Image.transform ).

% Здесь оцененные коэффициенты: только аффинные / с дисторсией
\begin{table}[H]
\centering
\caption{Эксперимент 1. Параметры связи координат и надежность модели ($\tau$)}
\label{table:coeffs_reliability}

\begin{tabular}{|l|l|l|l|l|l|}
\hline
\multicolumn{6}{|l|}{Только аффинное преобразование}\\ \hline
a     & b     & c       & d     & e      & f     \\ \hline
0.998 & 0.014 & -0.009  & 0.996 & 234 & 55 \\ \hline

$\eps_1, 10^{-8}$ & $\eps_2, 10^{-8}$ & $\mu_1, 10^{-14}$ & $\mu_2, 10^{-14}$ & $\tau$ & \\ \hline
-                 & -                 & -                 & -                 & 0.08   & \\ \hline

\hline
\multicolumn{6}{|l|}{Учет дисторсий 3го порядка}\\ \hline
a     & b     & c      & d     & e   & f  \\ \hline
0.995 & 0.014 & -0.010 & 0.995 & 234 & 56 \\ \hline

$\eps_1, 10^{-8}$ & $\eps_2, 10^{-8}$ & $\mu_1, 10^{-14}$ & $\mu_2, 10^{-14}$ & $\tau$ & \\ \hline
3.38              & 4.03              & -                 & -                 & 0.32   & \\ \hline

\hline
\multicolumn{6}{|l|}{Учет дисторсий 3 и 5 порядка}\\ \hline
a     & b     & c      & d     & e   & f  \\ \hline
0.988 & 0.014 & -0.010 & 0.988 & 245 & 64 \\ \hline

$\eps_1, 10^{-8}$ & $\eps_2, 10^{-8}$ & $\mu_1, 10^{-14}$ & $\mu_2, 10^{-14}$ & $\tau$ & \\ \hline
0.93              & 5.42              & 6.91              & -2.12             & 0.19   & \\ \hline
\end{tabular}
\end{table}

В таблице \ref{table:coeffs_reliability} можно заметить, что коэффициенты $a, b, c, d$ матрицы поворота близки к аналогичным от единичной матрицы. Значения дисторсных коэффициентов $\eps_1, \eps_2, \mu_1, \mu_2$ --- много меньше 1. Коэффициенты для дисторсии 5-го порядка имеют порядок $10^{-14}$, что близко к границе точности чисел типа double(число с плавающей запятой двойной точности, 64 бит) --- $10^{-15}$, которые использовалось в программе. Поэтому, нельзя исключать влияние ошибок машинного округления для модели D35.

Также на рисунках \ref{fig:align_a}, \ref{fig:align_d3}, \ref{fig:align_d35} можно увидеть итоговое положение звезд после процедуры виртуальной юстировки изображений. Можно заметить, что модели хорошо совмещают точки в центральной части. Однако, для совмещения периферийных звезд этих моделей оказывается недостаточно.

\begin{figure}[H]
  \centering
  \includegraphics[width=0.8\linewidth]{images/mag_diss/20161122-191517-359_000.png}
  \caption{Совмещение звезд. A: Только аффинное преобразование}
  \label{fig:align_a}
\end{figure}%

\begin{figure}[H]
  \centering
  \includegraphics[width=0.8\linewidth]{images/mag_diss/20161122-191517-359_030.png}
  \caption{Совмещение звезд. D3: С учетом дисторсии 3го порядка}
  \label{fig:align_d3}
\end{figure}

\begin{figure}[H]
  \centering
  \includegraphics[width=0.8\linewidth]{images/mag_diss/20161122-191517-359_035.png}
  \caption{Совмещение звезд. D35: С учетом дисторсии 3го, 5го порядков}
  \label{fig:align_d35}
\end{figure}

% % Результаты
% Можно заметить (иллюстрации на Рис. (\ref{fig:stars_center}, \ref{fig:stars_edge2}, \ref{fig:stars_edge3}) и Табл. \ref{tab:aff_dist_coeffs}), что:
% \begin{itemize}
%     \item Вблизи центра совмещение стало немного хуже, на краях --- значительно лучше.
%     \item Матрица поворота $ \begin{pmatrix} a & b \\ c & d \end{pmatrix} \approx \begin{pmatrix} 1 & 0 \\ 0 & 1 \end{pmatrix},$ т.е. камеры уже довольно хорошо съюстированы вручную.
%     \item $\eps_1 \approx \eps_2$, т.е. значения коэффициентов дисторсии на двух камерах согласуются.
% \end{itemize}

%
%%%%%%%%%%%
%

\newpage
\subsection{Эксперимент 2. Зависимость надежности модели от размера области выбора звезд для юстировки}
% % Здесь краткое описание эксперимента
В ходе этого эксперимента звезды, используемые для виртуальной юстировки выбирались следующим образом. Для вычисления параметров моделей A, D3, D35 использовались лишь те звезды, которые попадали в двух круги радиуса $R$, с центрами в соответствующих центральных точках левого и правого кадров. Затем вычислялось значение надежности для найденных коэффициентов по формуле:
\begin{equation}
  \label{eq:reliability}
  \tau_k(\xi) = \frac{ (2N - m_k) \sigma^2 }
                     { \| (I - A_k A_k^-) \xi \|^2 }
  ,
\end{equation}
где $k \in \{ A, D3, D35 \}$, $ m_k \in \{6, 8, 10\} $, $A_k$ --- матрица связи из \eqref{scheme_measure} для соответствующей модели $k$,
$\sigma$ --- корень из дисперсии вектора аддитивного шума.

Радиус центральной части увеличивался и вся процедура повторялась.

На \ref{fig:reliability_on_area} изображен итоговый график надежности для трех моделей. По горизонтальной оси отложено количество звезд, попавших в центральную область. Можно заметить, что надежность убывает с расширением центральной области. Следует отметить, что наилучшие результаты показывает модель, учитывающая дисторсию 3го порядка (D3).

% График \tau(n)
\begin{figure}[H]
  \centering
  \includegraphics[width=1\linewidth]{images/mag_diss/plot--reliability--on--stars_num.png}
  \caption{Зависимость надежности модели от размера области выбора звезд для юстировки. По горизонтальной оси отложено количество звезд, попадающих в центральную область (используемую для юстировки)}
  \label{fig:reliability_on_area}
\end{figure}

%
%%%%%%%%%%%
%

\newpage
\subsection{Эксперимент 3. Оценка значений высоты облаков при различных моделях юстировки}
% Здесь краткое описание эксперимента
В ходе эксперимента на предварительно съюстированной стереопаре решалась задача поиска фрагмента на изображении.
Затем, из найденного пиксельного сдвига между фрагментами вычислялось значение расстояния до объекта (облака) по формуле \eqref{eq:D_final}:
\begin{equation*}
  D = \frac{ B \xi_0 }
           { 2 tg \frac{\varphi_0}{2} | \xi_1 - \xi_2 | }
\end{equation*}

Ошибка высоты вычислялась по формуле
\begin{equation*}
    \Delta D =  \sqrt{ (\frac{\partial D}{\partial B}\Delta B)^2 + (\frac{\partial D}{\partial \varphi_0}\Delta \varphi_0)^2 + 2(\frac{\partial D}{\partial \xi_1}\Delta \xi_1)^2}
\end{equation*}

Задача поиска фрагмента на изображении решалась методами морфологического анализа изображений. При этом была использована оптимизация алгоритма поиска за счет вычисления сверток функций через быстрое преобразование Фурье. Таким образом, за счет увеличения количества используемой оперативной памяти, сложность алгоритма поиска уменьшилась с $O(N^2)$ до $O(N log(N))$, где $N$ --- характерный линейный размер изображения в пикселях.

\begin{figure}[H]
  \centering
  \includegraphics[width=\linewidth]{images/mag_diss/plot_20160831-154449-796_50.png}
  \caption{Сравнение результатов оценки высоты нижней границы облачности,
  полученных разработанной системой со значением, полученным лазерным
  дальномером (красная линия)}
  \label{fig:laser_cloud_height}
\end{figure}

На рисунке \ref{fig:laser_cloud_height} можно увидеть оценки высот облаков с их погрешностями для одной стереопары.
По вертикальной оси отложены значения высоты, по горизонтальной измерения упорядочены по интегральной яркости искомого фрагмента.
Красная горизонтальная линия представляет значение высоты, измеренное лазерным дальномером (ЛПР-1) в момент, близкий ко времени съемки стереопары.
Значения вблизи нуля высоты --- ошибки алгоритма поиска фрагмента.

\begin{table}[H]
\centering
\caption{Результаты оценки высоты облаков для различных фрагментов на 3-х стереопарах}
\label{tab:height}
\begin{tabular}{|l|l|l|l|l|}
\hline
\textnumero        & Модель A, м   & Модель D3, м & Модель D35, м & Значение дальномера, м \\ \hline
\multirow{3}{*}{1} & $2716 \pm 440$  & $2716 \pm 440$ & $4345 \pm 977$  & \multirow{3}{*}{1600 --- 1700}\\
                   & $2716 \pm 440$  & $3104 \pm 546$ & $3621 \pm 709$  &                               \\
                   & $2897 \pm 488$  & $2716 \pm 440$ & $3104 \pm 546$  &                               \\ \hline
%
\multirow{3}{*}{2} & $2556 \pm 399$  & $2716 \pm 440$ & $4828 \pm 118$2 & \multirow{3}{*}{2300 --- 3200}\\
                   & $3104 \pm 546$  & $3342 \pm 619$ & $4828 \pm 118$2 &                               \\
                   & $2716 \pm 440$  & $2897 \pm 488$ & $4345 \pm 977$  &                               \\ \hline
%
\multirow{6}{*}{3} & $1889 \pm 254$  & $1975 \pm 271$ & $3104 \pm 546$  & \multirow{6}{*}{2200}         \\
                   & $1889 \pm 254$  & $2069 \pm 290$ & $2414 \pm 365$  &                               \\
                   & $1889 \pm 254$  & $1810 \pm 240$ & $2414 \pm 365$  &                               \\
                   & $1889 \pm 254$  & $1889 \pm 254$ & $2172 \pm 312$  &                               \\
                   & $2069 \pm 290$  & $2287 \pm 337$ & $3342 \pm 619$  &                               \\
                   & $1975 \pm 271$  & $2069 \pm 290$ & $2414 \pm 365$  &                               \\ \hline
\end{tabular}
\end{table}

В таблице \ref{tab:height} приведены оценки высот с погрешностями, полученные с помощью разработанной системы. Для 3х стереопар (1, 2, 3) были оценены значения высот по нескольким фрагментам для 3х моделей (A, D3, D35) и проведено сравнение со значениями лазерного дальномера.
Лучшую согласованность с данными дальномера показали модели A, D3.

\begin{center}\section*{ЗАКЛЮЧЕНИЕ}\addcontentsline{toc}{section}{Заключение} \end{center}
В ходе проделанной работы удалось решить следующие задачи: была сформулирована и поставлена задача совмещения кадров стереопары с субпиксельной точностью, описаны примененные методы, разработаны и проведены эксперименты для проверки работоспособности методов и адекватности модели (написаны реализации алгоритмов в виде скриптов на Python).

Были получены следующие результаты: экспериментально показано наличие дисторсий и проверен метод избавления от них (совмещение звезд на краях больших изображений стало значительно лучше), подтверждена адекватность модели аффинной связи изображений стереопары при отсутствии дисторсий, метод субпиксельного совмещения был проверен и дал положительные результаты (адекватные оценки значений сдвигов $(1/2 - 1/32)$ пикселей для гладкой функции и $(1/2 - 1/4)$ для настоящего изображения) как на модельном (невязка $10-12\%$), так и на реальном изображении (невязка $2-32\%$), были предложены и проверены идеи для снижения количества вычислений --- деление спектра изображений, ограничение на область поиска фрагмента.

Таким образом, можно предположить, что вполне реально оценивать сдвиги с субпиксельной точностью с относительной погрешностью $\pm 50 \%$.

\addcontentsline{toc}{section}{Список использованных источников}
\renewcommand{\refname}{Список использованных источников}
\begin{thebibliography}{99}

\bibitem{art:refined} \emph{Chulichkov A.I., Medvedev A.P., Postylyakov O.V.} Refined optical model of stereophotography ground-based experiment for determination the cloud base height. Abs. SPIE Asia Pacific Remote Sensing, New Delhi, India
%, 4 - 7 April 2016
\bibitem{art:estimation1} \emph{Postylyakov O.V., Chulichkov A.I., Andreev M.S., Medvedev A.P.} Estimation of cloud height using ground-based stereophotography. Abs.7th Doas Workshop
%, 6–8 July 2015, Brussels, 96-97.
\bibitem{art:investigation} \emph{Postylyakov O.V., Andreev M.S., Elokhov A.S., Medvedev A.P., Chulichkov A.I., Borodko S.K., Borovski A.N., Bruchkouski I.I., Ivanov V.A., Krasouski A.N., Svetashev A.G.} Investigation of atmospheric composition using ground-based methods in cloudy conditions at Russian-Belorussian DOAS Network.  Abs. International Geographical Union (IGU) Regional Conference, Moscow
%August 17-21, 2015.
\bibitem{art:metod} \emph{Чуличков А.И., Андреев М.С., Постыляков О.В., Медведев А.П.} Метод определения нижней границы облачности по цифровой стереосъемке с поверхности земли
Тезисы XХI Международного симпозиума "Оптика атмосферы и океана. Физика атмосферы"
%, 22-26 июня 2015 года, Томск, №6862.
\bibitem{art:method} \emph{Chulichkov A.I., Andreev M.S., Emilenko A.S., Ivanov V.A., Medvedev A.P., Postylyakov O.V.} Method of estimation of cloud base height using ground-based digital stereophotography Proc. SPIE  9680, 21st International Symposium Atmospheric and Ocean Optics: Atmospheric Physics
%, 96804N (November 19, 2015
\bibitem{art:estimation2} \emph{Chulichkov A.I., Andreev M.S., Medvedev A.P., Postylyakov O.V.} Estimation of cloud base height using ground-based digital stereo photography. International symposium «Atmospheric radiation and dynamics» (ISARD – 2015).
%23 – 26 June 2015, Saint-Petersburg- Petrodvorets.
Theses. Pp. 86-87.
\bibitem{art:estimation3} \emph{Andreev M.S., Chulichkov A.I., Medvedev A.P., Postylyakov O.V.} Estimation of cloud base height using ground-based stereo photography: Method and first results. Proc. SPIE, Vol 9242, 924219, 2014, Pp. 924219-1—924219-7.
\bibitem{art:estimation4} \emph{Andreev M.S., Chulichkov A.I., Emilenko  A.S., Medvedev A.P., Postylyakov O.V.} Estimation of cloud height using ground-based stereophotography: Methods, error analysis and validation. Proc. SPIE 9259, Remote Sensing of the Atmosphere, Clouds, and Precipitation V, 92590N (November 20, 2014); Pp. 92590N-1--92590N-6


\bibitem{book:morpho} \emph{Пытьев Ю.П., Чуличков А.И.} Методы морфологического анализа изображений --- М.: ФИЗМАТЛИТ, 2010.
\bibitem{book:andreev_diplom} \emph{Андреев М.С.} Компьютерные методы оценивания высоты и скорости движения облаков по их изображениям, 2014 (дипломная работа).
\bibitem{book:exper} \emph{Пытьев Ю.П.} Математические методы интерпретации эксперимента --- М.: Высшая школа, 1989.
\bibitem{book:pytev_ivs} \emph{Пытьев Ю.П.} Методы математического моделирования измерительно-вычислительных систем. --- М.: ФИЗМАТЛИТ. 2012.
\bibitem{book:optic} \emph{Слюсарев Г.Г.} Методы расчета оптических систем, 2 изд., Л., 1969.
\bibitem{article:stereo} \emph{Jernej Mrovlje, Damir Vrancic} Distance measuring based on stereoscopic pictures, 2008.
\bibitem{book:fastfure} \emph{Нуссбаумер Г.} Быстрое преобразование Фурье и алгоритмы вычисления свёрток. --- М.: «Радио и связь». 1985.
\bibitem{book:chisl_met} \emph{Самарский А.А., Гулин А.В.} Численные методы: Учебное пособие для
вузов --- М.:~Наука, 1989.
%\bibitem{book:geometrypc} \emph{Hartley R., Zisserman A.} Multiple View Geometry in Computer Vision. Cambridge University Press 2004, 672 p.
%\bibitem{book:gons} \emph{Гонсалес Р., Вудс Р.} Цифровая обработка изображений. --- М.: Техносфера, 2005.
%\bibitem{book:gonsmatl} \emph{Гонсалес Р., Вудс Р., Эддинс С.} Цифровая обработка изображений в среде MATLAB. --- М.: Техносфера, 2006.



\end{thebibliography}



\end{document}
