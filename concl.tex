\begin{center}\section*{ЗАКЛЮЧЕНИЕ}\addcontentsline{toc}{section}{Заключение} \end{center}
В ходе проделанной работы удалось решить следующие задачи: была сформулирована и поставлена задача совмещения кадров стереопары с субпиксельной точностью, описаны примененные методы, разработаны и проведены эксперименты для проверки работоспособности методов и адекватности модели (написаны реализации алгоритмов в виде скриптов на Python).

Были получены следующие результаты: экспериментально показано наличие дисторсий и проверен метод избавления от них (совмещение звезд на краях больших изображений стало значительно лучше), подтверждена адекватность модели аффинной связи изображений стереопары при отсутствии дисторсий, метод субпиксельного совмещения был проверен и дал положительные результаты (адекватные оценки значений сдвигов $(1/2 - 1/32)$ пикселей для гладкой функции и $(1/2 - 1/4)$ для настоящего изображения) как на модельном (невязка $10-12\%$), так и на реальном изображении (невязка $2-32\%$), были предложены и проверены идеи для снижения количества вычислений --- деление спектра изображений, ограничение на область поиска фрагмента.

Таким образом, можно предположить, что вполне реально оценивать сдвиги с субпиксельной точностью с относительной погрешностью $\pm 50 \%$.
