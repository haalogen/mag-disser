\newpage
\begin{center}
  \section*{Выводы}
  \addcontentsline{toc}{section}{Выводы}
\end{center}

\begin{enumerate}
  \item Созданная система анализа изображений облаков способна эффективно определять
  параметры облачности, такие как высота и скорость.
  \item Проблему дисторсных искажений на изображениях с большим разрешением (3072x2304)
  удалось решить с помощью применения моделей  виртуальной юстировки, учитывающих
  дисторсионные искажения (D3, D35) и использования фрагментов для поиска из центральной
  части кадра. Модель D3 показывает лучший результат по критерию надежности.
  \item Оценки высоты облачности, полученные с помощью созданной системы, согласуются с
  измерениями лазерного дальномера. При расстоянии между камерами в 17 метров и
  точностью юстировки в 1 пиксель точность определения высоты облака меняется
  от 20 метров для высоты облака 200 метров (10\%) до 2900 метров для высоты облака 10000 метров (29\%).
  \item По результатам исследовательской работы были опубликованы работы:
  \cite{art:spie_2017}, \cite{art:spie_2016}, \cite{art:atm_optics_2016}.

\end{enumerate}



\newpage
\begin{center}
  \section*{Заключение}
  \addcontentsline{toc}{section}{Заключение}
\end{center}

В рамках данной работы было разработано математическое и программное обеспечение
информационной системы для оценки высоты и скорости движения облаков по их
изображениям высокого разрешения.

Была изучена и скомпонована в первой главе данной работы основная теория методов, на основании
которых была создана информационная система.

Была подтверждена работоспособность предложенного ранее метода определения расстояний
до облачности.

Были введены модели виртуальной юстировки изображений стереосистемы, учитывающие
дисторсионные искажения на снимках большого разрешения (D3, D35).

Адекватность моделей юстировки была оценена с помощью критерия надежности, имеющего
смысл вероятности ошибиться, отвергая модель, тем самым чем больше
надежность $\tau_k(\xi)$ тем выше согласие модели с измерением.
Наилучший результат показала модель, учитывающая дисторсию 3го порядка (D3).
В дальнейшем имеет смысл провести все вычисления на числах произвольной точности для более точного
вычисления коэффициентов дисторсных искажений и исключения влияния ошибки машинного округления.

Были определены условия на выбор фрагмента изображения облачности при которых
высота облачности может вычисляться с приемлемой точностью: при ошибке совмещения фрагментов
1 пиксель --- ошибка определения высоты лежит в промежутке от 10\% при высоте 200 м
до 29 \% при высоте 10 км.

Была разработана модель для оценки параметра скорости перемещения облаков с использованием
известной высоты облачности.

Были проведены измерения высот облачности на различных снимках облаков для всех трех
моделей виртуальной юстировки. Полученные оценки высот при сравнении с данными, полученными
с помощью лазерного дальномера (ЛПР-1), показали хорошую согласованность.