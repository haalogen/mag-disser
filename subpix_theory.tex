\subsection{Модель субпиксельного совмещения изображений}\label{subpixel}
Совмещения изображений с точностью до пикселя порой бывает недостаточно: чем больше дальность (высота) до облака, тем меньше расстояние (в пикселях) между найденными фрагментами совмещаемых изображений. Погрешность такой оценки может доходить до (и даже превышать) 100 \% от вычисленного значения дальности. Поэтому, актуальной является задача дальнейшего увеличения точности совмещения кадров стереопары.

Итак, необходимо построить модель двух кадров, смещенных друг относительно друга на крайне малое расстояние (менее 1 пикселя по каждой координате).

Будем считать, что функция яркости одного($\xi$) и другого($\eta$) кадров связаны малым сдвигом аругментов и линейным преобразованием. Также разложим функцию яркости первого изображения $f(x_i+\delta_x, y_j+\delta_y)$ в ряд Тейлора до линейных членов в точке $(x_i, y_j)$:
\begin{subequations} 
\begin{align}
    \xi_{ij} &= f(x_i+\delta_x, y_j+\delta_y) \approx f(x_i, y_j) 
    + f'_x(x_i, y_j)\delta_x + f'_y(x_i, y_j)\delta_y \label{eq:img_left} \\
    \eta_{ij} &= kf(x_i, y_j) + b , \label{eq:img_right}
\end{align}
\end{subequations}
где $i$ --- абсцисса пикселя, $j$ --- ордината пикселя,
$\delta_x$ --- сдвиг по $x$, $\delta_y$ --- сдвиг по $y$.

Тогда из \eqref{eq:img_left}
\begin{equation}
    f(x_i, y_j) = \xi_{ij} - f'_x(x_i, y_j)\delta_x - 
    f'_y(x_i, y_j)\delta_y,
\end{equation}
где $f'_x(x_i, y_j), \; f'_y(x_i, y_j)$ --- частные производные $f(\cdot, \cdot)$ по $x$ и $y$ соответственно (в точках $x_i, y_j$).

Из \eqref{eq:img_right} получим 
\begin{equation}
    f(x, y) = \frac{\eta_{ij}-b}{k} = \wtil{k}\eta_{ij} + \wtil{b},
\end{equation}
где $\wtil{k}=\frac{1}{k}, \; \wtil{b}=-\frac{b}{k}$.


Необходимо найти $(k, \; b, \; \delta_x, \; \delta_y)$ --- параметры преобразования одного изображения к другому. Для этого необходимо решить следующую задачу на минимум
\begin{equation}
    \sum_{i, j} \Big( \xi_{ij} - f'_x(x_i, y_j)\delta_x - f'_y(x_i, y_j)\delta_y - \wtil{k}\eta_{ij} - \wtil{b} \Big)^2 \rightarrow \min_{g}
\end{equation}
В последней формуле минимизируемый функционал является квадратичной
формой относительно неизвестных коэффициентов.

Оценим $f'_x(x_i, y_j)$ и $f'_y(x_i, y_j)$ из значений изображения $f(x_i+\delta_x, y_j+\delta_y)$, так как не знаем знаков $\delta_x$ и $\delta_y$. При работе с изображениями удобно индексировать точки, начиная с левого верхнего угла (точка $(0,0)$ --- в левом верхнем углу изображения). Мы будем поступать также:
\begin{equation*}
\begin{split}
    \textup{В неграничных точках:} \\
    f'_x(x_i, y_j) &= \frac{\xi_{i+1,j} - \xi_{i-1, j}}{2} \\
    f'_y(x_i, y_j) &= \frac{\xi_{i, j+1} - \xi_{i, j-1}}{2} \\
    \textup{В граничных точках:} \\
    f'_x(x_i, y_j) &= \xi_{i+1,j} - \xi_{i, j} \;\textup{(левая граница)} \\
    f'_x(x_i, y_j) &= \xi_{i,j} - \xi_{i-1, j} \;\textup{(правая граница)} \\
    f'_y(x_i, y_j) &= \xi_{i, j+1} - \xi_{i, j} \;\textup{(верхняя граница)} \\
    f'_y(x_i, y_j) &= \xi_{i, j} - \xi_{i, j-1} \;\textup{(нижняя граница)}
\end{split}
\end{equation*}
Эти формулы дают порядки аппроксимации по шагам равномерной сетки не ниже $O(h_x)$ и $O(h_y)$ \cite{book:chisl_met}.

Перепишем задачу на минимум в векторном виде:
\begin{equation} \label{eq:min_problem}
    \Big\| Ag - \xi \Big\|^2 \rightarrow \min_g
\end{equation}
Пусть $N$ --- число точек по $x$, $M$ --- число точек по $y$. Тогда $\xi = (\xi_{1,1} \dots \xi_{N,M}) \in R^{NM}$ --- вектор всех значений яркости первого изображения. Матрица $A \in R^{NM \times 4}$ заполнена следующим образом:
\begin{equation*}
\begin{split}
    A_{s, 1} &= \eta_{ij} \\
    A_{s, 2} &= 1 \\
    A_{s, 3} &= f'_x(x_i, y_j) \\
    A_{s, 4} &= f'_y(x_i, y_j) \\
    s &= i + jN = 1 \dots NM
\end{split}
\end{equation*}

Решением задачи \eqref{eq:min_problem} является вектор
\begin{equation*}
    g = \left( \begin{array}{c} \wtil{k} \\ \wtil{b} \\ \delta_x \\ \delta_y \end{array} \right) = A^- \xi = \Big( A^* A \Big)^{-1} A^* \;\xi
\end{equation*}

%
%%%%%%%%%%%
%
