\subsection{Юстировка камер}\label{cam_tuning}
В описанной в предыдущем разделе математической модели одним из требований является параллельность оптических осей обеих камер и одинаковый угол обзора.
К этому добавляется условие, что направления координатных осей $O_1X$ и $O_2X$, связанные с кадрами первой и второй камер, должны совпадать. В реальности выполнить эти требования довольно сложно. Однако, если иметь тестовые изображения одного и того же объекта, то можно преобразовать одно из изображений облака к виду, который оно имело бы при выполнении вышеназванных условий.

В настоящей работе, так же, как и в работе \cite{book:andreev_diplom}, в качестве тестовых объектов использовалось звездное небо. Звезды можно считать бесконечно удаленными точечными объектами, поэтому их изображения на фотокамерах, оси координат которых совмещены в соответствии с указанными выше требованиями, должны быть полностью идентичными, то есть координаты изображений одной и той же звезды на левой и правой камерах должны совпадать. Если этого не происходит, то камеры не юстированы должным образом.

Вместо совмещения осей координат поворотами камер найдем такое преобразование координат второй камеры, при котором изображения бесконечно удаленных объектов будут совпадать.

Однако при обработке изображений большого разрешения (3072x2304 в нашем случае) существенны искажения, связанные с дисторсиями оптических систем. Поэтому, избавимся в первую очередь от дисторсионных искажений.
%
%%%%%%%%%%%%%%%%%%%%%%%%%%%%%%%%%%%%%%%%%%%%%%
%

\subsubsection{Учет дисторсий фотокамер и аффинные преобразования}\label{distortion_affine}

\begin{figure}[h]
\center{ \includegraphics[width=1\linewidth]{stars_with_lines_600x400_mod.png} }
\caption{Прямые, проходящие через центр кадра, содержат изображения звезд первого и второго кадра. Это свидетельствует о наличии дисторсии.}
\label{img:starlines}
\end{figure}

При увеличении размера кадра до 3072х2304 пикселей начинает сказываться отличие используемой  модели геометрической оптики от реальности. Действительно, на рисунке \ref{img:starlines} представлены два кадра стереопары, наложенные друг на друга после аффинного преобразования; на одном из них звезды обозначены синим цветом, на другом – красным. Видно, что координаты звезд не совпадают, но если провести прямые через центр кадра и изображения звезд на первом и втором кадрах, получим систему прямых, к каждой из которых близки изображения одной и той же звезды каждого кадра. Это свидетельствует о наличии дисторсии \cite[стр.~73]{book:optic}. Учтем этот факт и уточним математическую модель связи между координатами звезд на правом и левом кадрах.

Пусть $(\xi_1, \eta_1)$ --- координаты точки объекта (звезды) на первом изображении, $(\xi_2, \eta_2)$ --- ее координаты на втором изображении.
Из-за дисторсии эти координаты отличаются от \emph{идеальных} координат $(x_1, y_1)$ и $(x_2, y_2)$, формируемых идеальными оптическими системами, причем таким образом, что изображение точки смещается к центру (либо от центра) кадра на величину, пропорциональную $l^3(x, y)$, где $l(x, y)$ --- расстояние от $(x, y)$ до центра кадра $(x_0, y_0)$ \cite[стр.~73]{book:optic}.

\begin{figure}[h]
\center{ \includegraphics[width=0.8\linewidth]{triangle_dist} }
\caption{Связь координат $(x, y)$ идеальной системы и координат $(\xi, \eta)$ системы с дисторсиями. Точка $B(x, y)$, точка $B'(\xi, \eta)$.}
\label{img:triangle_dist}
\end{figure}


Найдем связь между координатами $(x, y)$ идеального изображения звезды и ее координатами $(\xi, \eta)$ на реальном снимке. Для этого рассмотрим
два треугольника, $OBC$ и $OB'C'$, на рисунке \ref{img:triangle_dist}. Координаты точки В есть $(x, y)$, а точки $B'$ --- $(\xi, \eta)$, $O$ ---  центр кадра с координатами $(x_0, y_0)$.

Эти треугольники подобны, поэтому $ \frac{OC'}{OC} = \frac{OB'}{OB} $,
подставляя координаты точек $B, B', C $ и $ C'$, учитывая, что
$$
    |BB'| = \eps l^3 = \eps \left( \sqrt{ \distsqr } \right)^3 ,
$$
имеем
$$
    \frac{ (x-x_0) + (\xi-x) }{ x-x_0 } =
    \frac{ \Big(\distsqr \Big)^{1/2} + \eps \Big((x-x_0)^2 + (y-y_0)^2 \Big)^{3/2} }
    { \Big(\distsqr \Big)^{1/2} },
$$
откуда
$$
    \xi - x = \eps (x-x_0) \Big( \distsqr \Big), \\
$$
Аналогично, из равенства $ \frac{B'C'}{BC} = \frac{OB'}{OB} $
$$
    \eta - y = \eps (y-y_0) \Big( \distsqr \Big).
$$

Таким образом, искомая связь имеет вид
\begin{equation} \label{distort_transform}
\begin{split}
    x_i &= \xi_i - \eps_i z_x(x_i, y_i), \\
    y_i &= \eta_i - \eps_i z_y(x_i, y_i), \; i = 1,2 \;, \;
\textup{где} \\
    z_x(x, y) &= (x-x_0) \Big( \distsqr \Big), \\
    z_y(x, y) &= (y-y_0) \Big( \distsqr \Big),
\end{split}
\end{equation}
$(x_0, y_0)$ --- координаты центра кадра, а $\eps_1$ и $\eps_2$ --- коэффициенты дисторсий первого и второго кадра соответственно.

Координаты $(x_1, y_1)$ и $(x_2, y_2)$ звезд на первом и втором кадрах, сформированные идеальными видеокамерами, связаны аффинным преобразованием
\begin{equation} \label{pure_affine}
    \left( \begin{array}{c} x_2 \\ y_2 \end{array} \right) =
    \begin{pmatrix} a & b \\ c & d \end{pmatrix}
    \left( \begin{array}{c} x_1 \\ y_1 \end{array} \right) +
    \left( \begin{array}{c} e \\ f \end{array} \right)
\end{equation}

Подставив \eqref{distort_transform} в \eqref{pure_affine}, получим
\begin{equation*}
    \left( \begin{array}{c} \xi_2 - \eps_2 z_x(x_2, y_2) \\
    \eta_2 - \eps_2 z_y(x_2, y_2) \end{array} \right) =
    \begin{pmatrix} a & b \\ c & d \end{pmatrix}
    \left( \begin{array}{c} \xi_1 - \eps_1 z_x(x_1, y_1) \\
    \eta_1 - \eps_1 z_y(x_1, y_1) \end{array} \right) +
    \left( \begin{array}{c} e \\ f \end{array} \right)
\end{equation*}

Считая коэффициенты дисторсий $\eps_1$ и $\eps_2$ малыми по сравнению с единицей, коэффициенты $a$ и $d$ аффинных преобразований близкими к единице, $b$ и $c$ --- малыми по сравнению с единицей, и заменяя $z_x(x_i, y_i), \; z_y(x_i, y_i)$ на близкие к ним значения $z_x(\xi_i, \eta_i), \; z_y(\xi_i, \eta_i)$, запишем линеаризованную связь между наблюдаемыми величинами и неизвестными коэффициентами $g = (a, b, c, d, e, f, \eps_1, \eps_2)$:
\begin{equation}
  \begin{split}
  \left( \begin{array}{c} \xi_2^{(1)} \\ \eta_2^{(1)} \\ \vdots \\
  \xi_2^{(N)} \\ \eta_2^{(N)} \end{array} \right)
  %
  & = \begin{pmatrix} \xi_1^{(1)} & \eta_1^{(1)} & 0 & 0 & 1 & 0 &
  -z_x(\xi_1^{(1)}, \eta_1^{(1)}) & z_x(\xi_2^{(1)}, \eta_2^{(1)})\\
  0 & 0 & \xi_1^{(1)} & \eta_1^{(1)} & 0 & 1 &
  -z_y(\xi_1^{(1)}, \eta_1^{(1)}) & z_y(\xi_2^{(1)}, \eta_2^{(1)})\\
  %
  \vdots & \vdots & \vdots & \vdots & \vdots & \vdots & \vdots & \vdots \\
  %
  \xi_1^{(N)} & \eta_1^{(N)} & 0 & 0 & 1 & 0 &
  -z_x(\xi_1^{(N)}, \eta_1^{(N)}) & z_x(\xi_2^{(N)}, \eta_2^{(N)})\\
  0 & 0 & \xi_1^{(N)} & \eta_1^{(N)} & 0 & 1 &
  -z_y(\xi_1^{(N)}, \eta_1^{(N)}) & z_y(\xi_2^{(N)}, \eta_2^{(N)})\\
  \end{pmatrix}\\
  %
  & \times \begin{pmatrix} a & b & c & d & e & f & \eps_1 & \eps_2 \end{pmatrix}^T
  \end{split}
\end{equation}
где $N$ – число точечных объектов (звезд) на изображении.

Или в векторном виде, с учетом погрешности измерений,
\begin{equation} \label{scheme_measure}
    \xi = A g + \nu,
\end{equation}
где $g \in R^8$ --- вектор параметров преобразования, матрица $A$ размера $2N \times 8$ связывает вектор параметров $g$ с координатами точечных объектов на первом кадре. Вектор $\nu \in R^{2N}$ --- вектор погрешностей измерения координат точечных объектов. Матрицу $A$ считаем известной точно, вектор $g$ произволен, вектор $\nu$ --- случайный с нулевым математическим ожиданием и матрицей ковариаций $\Sigma$. В настоящей работе считалось, что координаты вектора $\nu$ некоррелированы и дисперсия координат равнялась квадрату удвоенного расстояния между соседними  пикселями цифрового изображения, так, что $\Sigma = \sigma^2 I$, где $I$ --- единичная матрица, $\sigma^2 = 4 \; (\textup{пиксель})^2$ (берем с запасом, чтобы покрыть также возможные погрешности в матрице $A$).

Как было показано в \cite[стр.~141]{book:pytev_ivs} и \cite{book:exper}, решением задачи редукции для данной модели является следующая оценка $\hat{f}$:
\begin{equation*}
\hat{f} = R\xi,
R = (A^* \Sigma^{-1} A)^{-1} A^* \Sigma^{-1} = A^- ,
\end{equation*}
т.к. $\Sigma$ --- диагональная матрица, $A^*$ --- транспонированная.

Из измерений \eqref{scheme_measure} оцениваются коэффициенты $g = (a, b, c, d, e, f, \eps_1, \eps_2)$, определяющие связь координат звезд на двух кадрах, далее юстировка осуществляется пересчетом изображений к виду, какой они имели бы, если бы дисторсии отсутствовали, а оптические оси камер и оси координат кадров первой и второй камер совпадали.

Пересчет координат производится по следующим формулам.
\begin{itemize}
    \item Пусть $f_1$ --- изображение на первом кадре, тогда при отсутствии дисторсий изображение того же объекта будет равно
\begin{equation} \label{eq:f1}
    \widetilde{f_1}(x, y) = f_1 \Big( x+\eps_1 z_x(x, y),\; y+\eps_1 z_y(x, y) \Big)
\end{equation}

    \item Пусть $f_2$ --- изображение на втором кадре, тогда при отсутствии дисторсий и рассогласований в направлениях оптической оси и осей координат видеокамер изображение того же объекта на втором кадре будет равно
\begin{equation} \label{eq:f2}
  \begin{split}
    & \wtil{f_2}(x, y) = f_2 \Big( \wtil{x}(x, y),\; \wtil{y}(x, y) \Big), \\
    %
    & \textup{где} \;
    %
    \left( \begin{array}{c} \wtil{x} \\ \wtil{y} \end{array} \right) =
    \begin{pmatrix} a & b \\ c & d \end{pmatrix}
    %
    \left( \begin{array}{c}
    \underbrace{x+\eps_2 z_x(x, y)}_{= \xi_2} \\ \underbrace{y+\eps_2 z_y(x, y)}_{= \eta_2}
    \end{array} \right) +
    \left( \begin{array}{c} e \\ f \end{array} \right)
  \end{split}
\end{equation}

\end{itemize}

Яркости пикселей изображений $\wtil{f_1}$ и $\wtil{f_2}$ получаются из \eqref{eq:f1}---\eqref{eq:f2} как значения функций $\wtil{f_1}(\cdot, \cdot)$ и $\wtil{f_2}(\cdot, \cdot)$  в узлах равномерной прямоугольной
сетки и вычисляются из яркостей пикселей изображений $f_1$ и $f_2$ с использованием интерполяции.


\subsubsection{Модель с учетом дисторсий 3го и 5го порядков}
Используя рассуждения, аналогичные предыдущей части, можно получить связь координат, учитывающую дисторсионные искажения 3го и 5го порядков. В этом случае решением задачи является вектор параметров $g = (a, b, c, d, e, f, \eps_1, \eps_2, \mu_1, \mu_2) \in R^{10}$.

