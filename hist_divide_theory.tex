\subsection{Деление спектра изображения}
При решении задачи совмещения кадров облаков удобно переходить от цветного (RGB) изображения к Grayscale-изображению (обычно 256 оттенков серого; <<0>> --- черный, <<255>> --- белый цвет). Тогда для каждой из 256 яркостей определена область данной яркости, задающая форму изображения облака. Для каждой из таких областей, в соответствии с предложенным методом и реализующим его вычислительным алгоритмом, следует вычислять интеграл свертки \eqref{eq:convolution}.  Однако, в получаемых изображениях облаков значительная часть Grayscale-спектра (обычно та, что ближе к <<0>>-черному цвету) отсутствует или очень слабо представлена. Поэтому, можно аппроксимировать исходное изображения с 256 уровнями яркости изображением с существенно меньшим числом уровней --- например, с 32 уровнями. Такую операцию будем называть \emph{делением спектра (яркости) изображения}.

Рассмотрим три метода такого деления спектра изображения на интервалы, соответствующие новым цветам (оттенкам):
\begin{enumerate}
    \item Равномерное деление --- самый простой для реализации и самый наивный метод деления, который, однако, дает адекватные результаты (например, при $256 \rightarrow 32$ градации).
    % здесь рисуночек спектра, поделенного равномерно
    
    \item Интегральный метод --- в основе лежит идея о делении спектра на участки, интегральные суммы которых примерно равны: 
$$
    I_1 \approx I_2 \approx \dots \approx I_n \approx \frac{I}{n}.
$$
Сложность алгоритма --- $O(N)$, где $N = 256$ оригинальных градаций.
    % здесь рисуночек спектра, поделенного интегрально
    
    \item Итерационный метод \cite[стр. 132]{book:morpho} --- итеративное деление на новые интервалы; нулевое приближение --- равномерное деление:
$$
    \wtil{c_j} = \textup{средняя яркость области $A_j$ на $m$-той итерации} =
    \underbrace{ \frac{1}{S_j} }_{ \textup{число пикселей} } \sum_{x_k \in A_j} f(x_k)
$$
Новые грaницы областей $A_1 \dots A_n$ на $m$-той итерации:
$$
    \underbrace{ 0 \quad;\quad \frac{\wtil{c}_1+\wtil{c}_2}{2} }_{A_1} \quad\dots\quad 
    \underbrace{ \frac{\wtil{c}_{n-1}+\wtil{c}_n} {2} \quad;\quad 255 }_{A_n}
$$
\end{enumerate}

В качестве критерия остановки итераций можно следить за изменением невязки между нашей кусочно-постоянной аппроксимацией и изначальным изображением $f(\cdot)$:
$$
    \sum_{x \in X} \Big( f(x) - \sum_{i} \wtil{c}_i \chi_i(x) \Big)^2 \rightarrow \min_{\{A_j\}}
$$

Результаты работы трех алгоритмов можно сравнить на Рис. \ref{fig:hist_divider}.

\begin{figure}[H]
\begin{subfigure}{\linewidth}
  \centering
  \includegraphics[height=0.265\textheight]{images/hist_divider/8col_iter0.png}
  \caption{Равномерное деление}
  \label{fig:uniform_div}
\end{subfigure}
\\
\begin{subfigure}{\linewidth}
  \centering
  \includegraphics[height=0.3\textheight]{images/hist_divider/8cols_integral.png}
  \caption{Интегральный метод}
  \label{fig:integral_div}
\end{subfigure}
\\
\begin{subfigure}{\linewidth}
  \centering
  \includegraphics[height=0.265\textheight]{images/hist_divider/8col_iter20.png}
  \caption{Итерационный метод}
  \label{fig:iterative_div}
\end{subfigure}
\caption{Деление спектра изображения облака тремя методами}
\label{fig:hist_divider}
\end{figure}