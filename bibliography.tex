\addcontentsline{toc}{section}{Список использованных источников}
\renewcommand{\refname}{Список использованных источников}
\begin{thebibliography}{99}

\bibitem{art:refined} \emph{Chulichkov A.I., Medvedev A.P., Postylyakov O.V.} Refined optical model of stereophotography ground-based experiment for determination the cloud base height. Abs. SPIE Asia Pacific Remote Sensing, New Delhi, India
%, 4 - 7 April 2016
\bibitem{art:estimation1} \emph{Postylyakov O.V., Chulichkov A.I., Andreev M.S., Medvedev A.P.} Estimation of cloud height using ground-based stereophotography. Abs.7th Doas Workshop
%, 6–8 July 2015, Brussels, 96-97.
\bibitem{art:investigation} \emph{Postylyakov O.V., Andreev M.S., Elokhov A.S., Medvedev A.P., Chulichkov A.I., Borodko S.K., Borovski A.N., Bruchkouski I.I., Ivanov V.A., Krasouski A.N., Svetashev A.G.} Investigation of atmospheric composition using ground-based methods in cloudy conditions at Russian-Belorussian DOAS Network.  Abs. International Geographical Union (IGU) Regional Conference, Moscow
%August 17-21, 2015.
\bibitem{art:metod} \emph{Чуличков А.И., Андреев М.С., Постыляков О.В., Медведев А.П.} Метод определения нижней границы облачности по цифровой стереосъемке с поверхности земли
Тезисы XХI Международного симпозиума "Оптика атмосферы и океана. Физика атмосферы"
%, 22-26 июня 2015 года, Томск, №6862.
\bibitem{art:method} \emph{Chulichkov A.I., Andreev M.S., Emilenko A.S., Ivanov V.A., Medvedev A.P., Postylyakov O.V.} Method of estimation of cloud base height using ground-based digital stereophotography Proc. SPIE  9680, 21st International Symposium Atmospheric and Ocean Optics: Atmospheric Physics
%, 96804N (November 19, 2015
\bibitem{art:estimation2} \emph{Chulichkov A.I., Andreev M.S., Medvedev A.P., Postylyakov O.V.} Estimation of cloud base height using ground-based digital stereo photography. International symposium «Atmospheric radiation and dynamics» (ISARD – 2015).
%23 – 26 June 2015, Saint-Petersburg- Petrodvorets.
Theses. Pp. 86-87.
\bibitem{art:estimation3} \emph{Andreev M.S., Chulichkov A.I., Medvedev A.P., Postylyakov O.V.} Estimation of cloud base height using ground-based stereo photography: Method and first results. Proc. SPIE, Vol 9242, 924219, 2014, Pp. 924219-1—924219-7.
\bibitem{art:estimation4} \emph{Andreev M.S., Chulichkov A.I., Emilenko  A.S., Medvedev A.P., Postylyakov O.V.} Estimation of cloud height using ground-based stereophotography: Methods, error analysis and validation. Proc. SPIE 9259, Remote Sensing of the Atmosphere, Clouds, and Precipitation V, 92590N (November 20, 2014); Pp. 92590N-1--92590N-6


\bibitem{book:morpho} \emph{Пытьев Ю.П., Чуличков А.И.} Методы морфологического анализа изображений --- М.: ФИЗМАТЛИТ, 2010.
\bibitem{book:andreev_diplom} \emph{Андреев М.С.} Компьютерные методы оценивания высоты и скорости движения облаков по их изображениям, 2014 (дипломная работа).
\bibitem{book:exper} \emph{Пытьев Ю.П.} Математические методы интерпретации эксперимента --- М.: Высшая школа, 1989.
\bibitem{book:pytev_ivs} \emph{Пытьев Ю.П.} Методы математического моделирования измерительно-вычислительных систем. --- М.: ФИЗМАТЛИТ. 2012.
\bibitem{book:optic} \emph{Слюсарев Г.Г.} Методы расчета оптических систем, 2 изд., Л., 1969.
\bibitem{article:stereo} \emph{Jernej Mrovlje, Damir Vrancic} Distance measuring based on stereoscopic pictures, 2008.
\bibitem{book:fastfure} \emph{Нуссбаумер Г.} Быстрое преобразование Фурье и алгоритмы вычисления свёрток. --- М.: «Радио и связь». 1985.
\bibitem{book:chisl_met} \emph{Самарский А.А., Гулин А.В.} Численные методы: Учебное пособие для
вузов --- М.:~Наука, 1989.

\bibitem{book:hartley_geometry} \emph{Hartley R., Zisserman A.} Multiple View Geometry in Computer Vision. Cambridge University Press 2004, 672 p.
\bibitem{book:forsyth_geometry} \emph{Форсайт Д., Понс Ж.} Компьютерное зрение. Современный подход. М.: Вильямс, 2004. — 928 с.
\bibitem{book:gonsalez} \emph{Гонсалес Р., Вудс Р.} Цифровая обработка изображений. --- М.: Техносфера, 2005.
\bibitem{art:zuev} \emph{Зуев, С. В., Красненко, Н. П., Левикин, В. А.} (2014). Телевизионный измеритель характеристик облачности. Доклады Томского государственного университета систем управления и радиоэлектроники, (1 (31)), 54-59.

КиберЛенинка: https://cyberleninka.ru/article/n/televizionnyy-izmeritel-harakteristik-oblachnosti
%\bibitem{book:gonsmatl} \emph{Гонсалес Р., Вудс Р., Эддинс С.} Цифровая обработка изображений в среде MATLAB. --- М.: Техносфера, 2006.

\bibitem{art:spie_2017} \emph{Chulichkov A.I., Nikitin S.V., Emilenko A.S., Medvedev A.P., Postylyakov O.V.}
Selection of optical model of stereophotography experiment for
determination the cloud base height as a problem of testing of statistical
hypotheses. - В сборнике Proc. SPIE, серия Remote Sensing of Clouds and
the Atmosphere XXII, Warsaw, Poland, 2017г., том 10424, с. 1-11.

\bibitem{art:spie_2016} \emph{Chulichkov A.I., Nikitin S.V., Medvedev A.P., Postylyakov O.V.}
Stereoscopic ground-based determination of the cloud base height: camera
position adjusting with account for lens distortion. - В сборнике Proc. of SPIE,
серия AOO16 - Optical Investigation of Atmosphere and Ocean, 2016, Томск,
том 10035, с. 100353B-1-100353B-10

\bibitem{art:atm_optics_2016} \emph{Чуличков А.И., Никитин С.В., Медведев А.П., Постыляков О.В.}
Уточненная оптическая модель наземного эксперимента по
определению высоты нижней границы облачности по её стереосъемке.
Материалы XХII Международного симпозиума "Оптика атмосферы и
океана», Томск, 2016.




\end{thebibliography}

