\section*{Обзор литературы}
\addcontentsline{toc}{section}{Обзор литературы}

\subsection*{Фотограмметрия}

В работах
\cite{book:hartley_geometry},
\cite{book:forsyth_geometry}
можно ознакомиться с теоретическими основами фотограмметрии.

В статье
\cite{article:stereo}
предложена геометрическая схема определения дистанции с помощью стереосистемы из 2х камер,
используемая в данной работе.

В статье
\cite{art:zuev} представлен измеритель характеристик облачности, позволяющий определять
высоту, направление и скорость движения нижней облачности по ее разномасштабным изображениям,
а также общий балл облачности по панорамным изображениям всего небосвода.


Также стоит отметить модуль Camera Calibration and 3D Reconstruction (OpenCV)
для калибровки стереосистем, особое внимание стоит обратить на функции:
\begin{itemize}
  \item[--] calibrateCamera(...) -- оценка параметров камеры.
  \item[--] stereoCalibrate(...) -- виртуальное совмещение 2х камер стереопары.
\end{itemize}


\subsection*{Поиск фрагмента изображения}
Алгоритмы SIFT, SURF, ORB (есть реализации в OpenCV)
--- основаны на дескрипторах ключевых точек изображения, при этом они имеют
довольно высокую вычислительную сложность.
Однако эти алгоритмы обладают многими ценными свойствами, такими как:
робастность к изменению масштаба, поворотам в 3D, дисторсиям, шуму,
искажениям, изменению освещения сцены.

В монографии \cite{book:morpho} можно ознакомиться с анализом изображений с помощью морфологии Пытьева.
Методы морфологического анализа Пытьева используются в данной работе для решения задачи
поиска фрагмента на изображении.

Также может быть полезно ознакомиться с
\cite{book:gonsalez}, где обсуждаются вопросы распознавания образов, морфологического анализа изображения.