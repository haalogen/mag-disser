\begin{center}\section{Математические модели  и методы анализа  изображений}\end{center}
\subsection{Геометрическая модель эксперимента}\label{geometry}

\subsubsection{Математическая модель стереопары изображений}
Для оценки расстояния до объекта по стереопаре его изображений опишем сначала математическую модель стереопары изображений при условии, что обе фотокамеры имеют одни и те же параметры (увеличение) и технические характеристики, а их оптические оси направлены в зенит. Согласно \cite{book:andreev_diplom} и \cite{article:stereo}, мы можем использовать следующую модель.

\begin{figure}
\center{\includegraphics[width=1\linewidth]{scheme}}
\caption{Схема формирования изображения точки облака двумя фотокамерами}
\label{img:scheme}
\end{figure}


С каждой фотокамерой свяжем декартову систему координат, в которой ось $O_iZ$ направлена вертикально вверх и совпадает с ее оптической осью, горизонтальная ось $O_iX$ проходит через центры камеры и направлена от камеры $S_L$ к камере $S_R, \; i=1,2$. Пусть облако расположено на высоте $D$, расстояние между фотокамерами по горизонтальной оси равно $B$ (размер \emph{базы}). Пусть $x_0$ - горизонтальный размер пространственной области на высоте $D$, изображаемой фотокамерой, $\varphi_0$ - угол обзора камеры (Рис.~\ref{img:scheme}). Тогда
\begin{equation}\label{eq:D_first}
    D = \frac{x_0}{2tg(\frac{\varphi_0}{2})}
\end{equation}


Выберем точку облака (для наглядности на Рис.~\ref{img:scheme} эта точка лежит в плоскости, проходящей через оптические оси обеих камер). Обозначим $x_1$ и $x_2$ координаты выбранной точки облака вдоль осей $O_1X$ и $O_2X$ в системе координат первой и второй камер соответственно. Для точки облака, расположенной в плоскости оптических осей, выполнено равенство  $|x_1 - x_2| = B$. Поэтому вместо \eqref{eq:D_first} имеем:
\begin{equation}\label{eq:D_second}
    D = \frac{Bx_0}{2tg(\frac{\varphi0}{2})|x_1 - x_2|}
\end{equation}


В формуле (\ref{eq:D_second}) наблюдаемыми являются параметры $B$ (стереобаза), $\varphi_0$ (угол обзора фотокамеры) и отношение $\frac{x_0}{|x_1-x_2|}$ - оно равно отношению $\frac{\xi_0}{|\xi_1 - \xi_2|}$ размера кадра $\xi_0$ к модулю разности координат изображений
выбранной точки облака на первом и втором кадре $|\xi_1 - \xi_2|$. Итак, формула для вычисления высоты облака есть
\begin{equation}\label{eq:D_final}
D = \frac{B\xi_0}{2tg\frac{\varphi_0}{2}|\xi_1 - \xi_2|}
\end{equation}

%
%%%%%%%%%%%%%%%%%%%%%%%%%%%%%%%%%%%%%%%%%%%%%%%%%%%%%%%%%%%%%%%%%%
%

\subsubsection{Оценка погрешности определения расстояния до облака}

Погрешность определения высоты облака с помощью формулы (\ref{eq:D_final}) оценим по формуле для погрешности косвенных измерений:
\begin{equation*}
    \Delta D =  \sqrt{ (\frac{\partial D}{\partial B}\Delta B)^2 + (\frac{\partial D}{\partial \xi_0}\Delta \xi_0)^2 + (\frac{\partial D}{\partial \varphi_0}\Delta \varphi_0)^2 + (\frac{\partial D}{\partial \xi_1}\Delta \xi_1)^2 + (\frac{\partial D}{\partial \xi_2}\Delta \xi_2)^2 }
\end{equation*}
Заметим, что $\Delta \xi_0 = 0$, т.к. размеры изображения мы всегда знаем точно. Также, обычно $\Delta \xi_1 = \Delta \xi_2$, поэтому формула упрощается:
\begin{equation*}
    \Delta D =  \sqrt{ (\frac{\partial D}{\partial B}\Delta B)^2 + (\frac{\partial D}{\partial \varphi_0}\Delta \varphi_0)^2 + 2(\frac{\partial D}{\partial \xi_1}\Delta \xi_1)^2}
\end{equation*}

Если проанализировать, какие слагаемые дают наибольший вклад в погрешность, то заметим, что подавляющую часть вносят погрешности  $(\frac{\partial D}{\partial \xi_i}\Delta \xi_i), \; i=1,2.$  Существенно уменьшить эти погрешности можно:
\begin{itemize}
\item увеличивая размер изображений;
\item повышая точность определения координат $\xi_i$ (<<субпиксельная точность>>).
\end{itemize}
И то и другое приведет к росту числа вычислений. Поэтому целесообразно разработать и применить методы к их сокращению:
\begin{itemize}
\item деление спектра Grayscale-изображения (яркость $0 \dots 255$) на $m < 256$ промежутков, соответствующих новым (средним по промежутку) яркостям;
\item ограничение области поиска фрагмента, ближайшего к эталону, из соображений характерных высот облаков (от 200 м до 10 км).
\end{itemize}

%
%%%%%%%%%%%%%%%%%%%%%%%%%%%%%%%%%%%%%%%%%%%%%%%%%%%%%%%%%%%%%%%%%%
%

\subsubsection{Оценка скорости движения облаков}
Рассмотрим два снимка облака сделанные одной и той же камерой с разницей во
времени $\Delta t$. Пусть перемещение фрагмента облака в плоскости соответствует
сдвигам $\Delta x, \Delta y$ (в метрах) и пиксельным сдвигам
$\Delta \xi, \Delta \eta$ изображений. Введем обозначения:
\begin{align*}
  R &= \sqrt{\Delta x^2 + \Delta y^2}\\
  r &= \sqrt{\Delta \xi^2 + \Delta \eta^2}
\end{align*}
Считаем, что высота облака $D$ за время $\Delta t$ изменилась не существенно.
Тогда, из планиметрических соображений можно записать отношение:
\begin{equation}
  \frac{D}{f} = \frac{
    \sqrt{\Delta x^2 + \Delta y^2}
  }{
    \sqrt{\Delta \xi^2 + \Delta \eta^2}
  } = \frac{R}{r}
\end{equation}
Здесь $f$ --- фокусное расстояние камеры (порядка 50 мм), $D$ --- высота
облачности.

Тогда скорость облака можем оценить как
\begin{equation}
  v = \frac{R}{\Delta t} =  \frac{D \cdot r}{\Delta t \cdot f}
\end{equation}

Значения скорости различных облаков могут варьироваться в пределах от 10 до
200 м/с.