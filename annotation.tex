\documentclass[12pt,a4paper]{article}
\usepackage{amsmath}
\usepackage{amsfonts}
\usepackage{amssymb}


\usepackage{cmap}       % Поддержка поиска русских слов в PDF (pdflatex)

\usepackage{indentfirst}% Красная строка в первом абзаце

\usepackage{mathtext}   % если нужны русские буквы в формулах
                        % перенос слов с тире
\lccode`\-=`\-
\defaulthyphenchar=127
            % Выбор языка и кодировки
\usepackage[T2A]{fontenc}
\usepackage[utf8]{inputenc}
\usepackage[english, russian]{babel}
\usepackage[affil-it]{authblk}

\title{Морфологическое оценивание относительного сдвига двух изображений с субпиксельной точностью}
\author{Никитин Станислав}
\affil{ Физический факультет МГУ им. М.В.~Ломоносова\\
Кафедра математического моделирования и информатики }
\date{}

\begin{document}
\maketitle
\pagenumbering{gobble}
\section*{Аннотация}
В данной работе рассматривается проблема совмещения пары изображений облаков, одновременно снятых с двух камер, оптические оси которых направлены в зенит с целью определения дальности до них по стреопарам. В работе поставлена задача определения расстояния до облаков. Показана необходимость максимально точного совмещения изображения облаков на двух кадрах стереопары. Обосновано применение для этой цели методов морфологического анализа изображений, разработанных в работах Ю.П.Пытьева. Для юстировки изображений стереопары используется метод, учитывающий аффинные преобразования, возникающие за счет несогласованности оптических осей фотокамер и осей координат кадров, а также дисторсии оптических систем. В работе также предложены методы сокращения времени вычислений: используется метод быстрого преобразования Фурье для вычисления морфологической близости изображений стереопары, предложен метод оптимального сокращения числа уровней яркости изображения, метод использования априорных ограничений на расстояния до облаков. Работоспособность методов демонстрируется в вычислительных экспериментах.
\end{document}